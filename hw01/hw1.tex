\documentclass{amsart}

\usepackage{enumitem}

\theoremstyle{definition}
\newtheorem{exercise}{Exercise}

\newcommand{\N}{\mathbb{N}}
\newcommand{\Z}{\mathbb{Z}}
\newcommand{\Q}{\mathbb{Q}}
\newcommand{\R}{\mathbb{R}}
\newcommand{\abs}[1]{|#1|}
\newcommand{\st}{\mathrel{:}}

\title{MATH 355: Homework 1}
\author{Alexander Lee}

\begin{document}

\maketitle

\begin{exercise}[1.2.2]
  Suppose towards a contradiction that there is a rational number $r \in \Q$
  satisfying $2^r = 3$. Since $r \in \Q$, we can write $r = \frac{p}{q}$ for
  some $p, q \in \Z$ with $q \neq 0$. Thus, we have $2^\frac{p}{q} = 3
  \Rightarrow 2^p = 3^q$.

  Suppose $r > 0$. We can assume that $p$ and $q$ have no common factors, so
  $p, q \in \Z^+$. Since $2^p = 2 \cdot 2^{p-1}$ and $2^{p-1} \in \Z$, $2^p$ is
  even. As such, $3^q$ is even as well. However, this is a contradiction since
  $3^q$ is odd because the product of odd numbers is odd.

  Next, suppose $r < 0$. Without loss of generality, suppose $p \in \Z^-$ and $q
  \in \Z^+$. Since $2^\frac{p}{q} < 1$, we have a contradiction since we assumed
  that $2^\frac{p}{q} = 3$.

  Finally, suppose $r = 0$. Then, $2^0 = 1 \neq 3$, which is a contradiction.
\end{exercise}

\begin{exercise}[1.2.3]
  \begin{enumerate}[label={(\alph*)}]
    \item False. Consider infinite set of the form $A_n = [0, \frac{1}{n}]$ for
      $n \in \N$. Our definition of $A_n$ satisfies $A_1 \supseteq A_2 \supseteq
      A_3 \supseteq A_4 \cdots$. However, notice that $\bigcap_{n=1}^{\infty}
      A_n = \{0\}$, which is not an infinite set.
    \item True.
    \item False. Let $A = \{0\}$, $B = \{0, 1\}$, and $C = \{2, 3\}$. Then,
      \[
        A \cap (B \cup C) = \{0\} \cap (\{0, 1\} \cup \{2, 3\}) = \{0\},
      \]
      but
      \[
        (A \cap B) \cup C = (\{0\} \cap \{0, 1\}) \cup \{2, 3\} = \{0, 2, 3\}.
      \]
      Here, $A \cap (B \cup C) \neq (A \cap B) \cup C$.
    \item True.
    \item True.
  \end{enumerate}
\end{exercise}

\begin{exercise}[1.2.6]
  \begin{enumerate}[label={(\alph*)}]
    \item Suppose $a, b \in \R$ where $a, b > 0$. We have that $\abs{a + b} = a
      + b = \abs{a} + \abs{b}$.

      Next, suppose that $a, b < 0$. We have that $\abs{a + b} = -(a + b) = (-a)
      + (-b) = \abs{a} + \abs{b}$.
    \item Given $a, b \in \R$,
      \begin{align*}
        {(a + b)}^2 &= a^2 + 2ab + b^2 \\
        &\le \abs{a}^2 + 2\abs{a}\abs{b} + \abs{b}^2 \\
        &= {(\abs{a} + \abs{b})}^2.
      \end{align*}
    \item Given $a, b, c, d \in \R$,
      \begin{align*}
        \abs{a - b} &= \abs{(a - c) + (c - d) + (d - b)} \\
        &\le \abs{a - c} + \abs{(c - d) + (d - b)}\ \text{(by the triangle
        inequality)} \\
        &\le \abs{a - c} + \abs{c - d} + \abs{d - b}\ \text{(by the triangle
        inequality)}.
      \end{align*}
    \item Given $a, b \in \R$,
      \begin{align*}
        \abs{\abs{a} - \abs{b}} &= \abs{\abs{a - b + b} - \abs{b}} \\
        &\le \abs{\abs{a - b} + \abs{b} - \abs{b}}\ \text{(by the triangle
        inequality)} \\
        &= \abs{\abs{a - b}} \\
        &= \abs{a - b}.
      \end{align*}
  \end{enumerate}
\end{exercise}

\begin{exercise}[1.2.8]
  \begin{enumerate}[label={(\alph*)}]
    \item Let $f : \N \rightarrow \N$ be defined by $f(n) = n + 1$, $n \in \N$.
    \item Let $f : \N \rightarrow \N$ be defined by $f(n) = \abs{n}$, $n \in \N$.
    \item Let $f : \N \rightarrow \Z$ be defined by $f(n) =
      \begin{cases}
        \frac{n - 1}{2} &\, n\ \text{is odd} \\
        \frac{-n}{2} &\, n\ \text{is even}
      \end{cases}$,
      $n \in \N$.
  \end{enumerate}
\end{exercise}

\begin{exercise}[1.2.10]
  \begin{enumerate}[label={(\alph*)}]
    \item False. Let $a = 1$, $b = 1$, and $\epsilon = 0.5$. We have that $1 < 1
      + 0.5 = 1.5$, but it is not true that $a < b$ since $a = 1 = b$.
    \item False. The same counterexample as in part (a) can be used to show the
      statement is false.
    \item False. Let $a = 2$, $b = 1$, and $\epsilon = 2$. We have that $2 < 1 +
      2 = 3$, but $a > b$ since $2 > 1$.
  \end{enumerate}
\end{exercise}

\begin{exercise}[1.3.2]
  \begin{enumerate}[label={(\alph*)}]
    \item Let $B = \{0\}$. $\inf(B) = 0 = \sup(B)$. Thus, $\inf(B) \ge \sup(B)$.
    \item Impossible.
    \item Let $B$ be a bounded subset of $\Q$ where $B = \{r \in \Q \st 0 < r
      \le 1 \}$. $\sup(B) = 1 \in B$, but $\inf(B) = 0 \notin B$.
  \end{enumerate}
\end{exercise}

\begin{exercise}[1.3.3]
  \begin{enumerate}[label={(\alph*)}]
    \item By definition, $\inf(A) \in B$ and $\inf(A) \ge b$ for all $b \in B$.
      Thus, $\inf(A)$ is the maximum of $B$, which implies that $\inf(A) =
      \sup(B)$.
    \item For every nonempty set $A$ of real numbers that is bounded below, we
      can define $B = \{b \in \R \st b\ \text{is a lower bound for $A$}\}$. By
      the Axiom of Completeness, we know that if $B$ is bounded above, $B$ has a
      least upper bound $\sup(B)$. From (a), we showed that $\sup(B) = \inf(A)$.
      Therefore, if $A$ is bounded below, $A$ has a greatest lower bound
      $\inf(A)$, so there is no need to assert this in the Axiom of
      Completeness.
  \end{enumerate}
\end{exercise}

\begin{exercise}[1.3.6]
  \begin{enumerate}[label={(\alph*)}]
    \item Given $a \in A$ and $b \in B$, consider
      \begin{align*}
        a + b &\le s + b\ \text{(since $s = \sup(A) \ge a$)} \\
        &\le s + t\ \text{(since $t = \sup(B) \ge b$)}.
      \end{align*}
      Therefore, $s + t$ is an upper bound for $A + B$.
    \item Given an arbitrary upper bound $u$ for  $A + B$, $a \in A$, and $b \in
      B$, we have $a + b \le u \Rightarrow b \le u - a$. Thus, $u - a$ is an
      upper bound for $B$. Since $t = \sup(B)$ is the least upper bound for $B$,
      $t \le u - a$.
    \item It follows from (b) that $a \le u - t$. Thus, $u - t$ is an upper
      bound for $A$. As such, $\sup(A) \le u - t \Rightarrow s \le u - t
      \Rightarrow s + t \le u$. Therefore, $\sup(A + B) = s + t$.
    \item Since $s = \sup(A)$ and $t = \sup(B)$, given an arbitrary $\epsilon >
      0$, $a \in A$, and $b \in B$, we know $s - \frac{\epsilon}{2} < a$ and $t
      - \frac{\epsilon}{2} < b$ by Lemma 1.3.8. Then,
      \begin{align*}
        (s - \frac{\epsilon}{2}) + (t - \frac{\epsilon}{2}) &< a + b \\
        s + t - \epsilon < a + b.
      \end{align*}
      By Lemma 1.3.8, $\sup(A + B) = s + t$.
  \end{enumerate}
\end{exercise}

\end{document}
