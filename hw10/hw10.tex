\documentclass{amsart}

\usepackage{enumitem}
\usepackage{mathtools}

\theoremstyle{definition}
\newtheorem{exercise}{Exercise}

\DeclarePairedDelimiter\abs{\lvert}{\rvert} % absolute value
% Swap the definition of \abs*, so that \abs
% resizes the size of the bars, and the starred version does not.
\makeatletter
\let\oldabs\abs%
\def\abs{\@ifstar{\oldabs}{\oldabs*}}
\makeatother

\newcommand{\N}{\mathbb{N}}
\newcommand{\Z}{\mathbb{Z}}
\newcommand{\Q}{\mathbb{Q}}
\newcommand{\I}{\mathbb{I}}
\newcommand{\R}{\mathbb{R}}
\newcommand{\card}[1]{\abs{#1}}
\newcommand{\st}{\mathrel{:}}

\title{MATH 355: Homework 10}
\author{Alexander Lee}

\begin{document}

\maketitle

\begin{exercise}[5.3.2]
  Suppose $f(a) = f(b)$ for some $a, b \in A$. Suppose towards a contradiction
  that $a \neq b$. Without loss of generality, suppose $a < b$. By the Mean
  Value Theorem, since $f : [a, b] \to \R$ is continuous on $[a, b]$ and
  differentiable on $(a, b)$, there exists a point $c \in (a, b)$ where
  \[
    f'(c) = \frac{f(b) - f(a)}{b - a}.
  \]
  However, since $f(a) = f(b)$ and $b \neq a$, we have that $f'(c) = \frac{0}{b
  - a} = 0$. This is a contradiction, since we assumed that $f'(x) \neq 0$ for
  all $x \in A$. Thus, it must be that $a = b$ and therefore $f$ is one-to-one
  on $A$.

  To show that the converse statement need not be true, consider the
  differentiable function $f(x) = x^3$ on the interval $A = (-1, 1)$. Here,
  $f$ is clearly one-to-one, but $f'(0) = 0$.
\end{exercise}

\begin{exercise}[5.3.3 skip (c)]
  \begin{enumerate}[label={(\alph*)}]
    \item Consider the function $f(x) = h(x) - x$ defined on the interval $[0,
      3]$. To argue that there exists a point $d \in [0, 3]$ where $h(d) = d$,
      we show that there exists a point $d \in [0, 3]$ where $f(d) = 0$. First,
      consider
      \[
        f(0) = h(0) - 0 = 1 - 0 = 1.
      \]
      Then, consider
      \[
        f(3) = h(3) - 3 = 2 - 3 = -1.
      \]
      Also notice that because $h$ is differentiable on $[0, 3]$, $h$ is also
      continuous on $[0, 3]$. It follows from the Algebraic Continuity Theorem
      that $f(x) = h(x) - x$ is also continuous on $[0, 3]$. Since $f$ is
      continuous on $[0, 3]$ and $f(3) = -1 < 0 < 1 = f(0)$, there exists a
      point $d \in (0, 3)$ where $f(d) = 0$ by the Intermediate Value Theorem.
      It follows that there exists a point $d \in [0, 3]$ where $h(d) = d$.
    \item Since $h$ is differentiable on $[0, 3]$, $h$ is thus continuous on
      $[0, 3]$ and differentiable on $(0, 3)$. Then, by the Mean Value Theorem,
      there exists a point $c \in (0, 3)$ where
      \[
        h'(c) = \frac{h(3) - h(0)}{3 - 0} = \frac{2 - 1}{3} = \frac{1}{3}.
      \]
  \end{enumerate}
\end{exercise}

\begin{exercise}[5.3.4]
  \begin{enumerate}[label={(\alph*)}]
    \item Because $f$ is differentiable on $A$, $f$ is also continuous on $A$.
      By the Characterizations on Continuity, since $(x_n) \to 0$, it follows
      that $f(x_n) \to f(0)$. Because $f(x_n) = 0$ for all $n \in \N$, it must
      be that $f(0) = 0$. Furthermore, since $f$ is differentiable at 0 (and
      $x_n \neq 0$), we have that
      \[
        f'(0) = \lim \frac{f(x_n) - f(0)}{x_n - 0} = \lim \frac{0 - 0}{x_n} =
        \lim \frac{0}{x_n} = 0.
      \]
    \item Observe that since $f_n(x_n) = 0$ for all $n \in \N$, we also have
      $f_n'(x) = 0$ for all $n \in \N$. Thus, given that $f$ is
      twice-differentiable at zero, we have that
      \[
        f''(0) = \lim \frac{f'(x_n) - f'(0)}{x_n - 0} = \lim \frac{0 - 0}{x_n} =
        \lim \frac{0}{x_n} = 0.
      \]
  \end{enumerate}
\end{exercise}

\begin{exercise}[6.2.1]
  \begin{enumerate}[label={(\alph*)}]
    \item The pointwise limit of $(f_n)$ for all $x \in (0, \infty)$ is $f(x) =
      \frac{1}{x}$.
    \item TODO
    \item TODO
    \item TODO
  \end{enumerate}
\end{exercise}

\begin{exercise}[6.2.2]
\end{exercise}

\end{document}
