\documentclass{amsart}

\usepackage{enumitem}
\usepackage{mathtools}

\theoremstyle{definition}
\newtheorem{exercise}{Exercise}

\newcommand{\N}{\mathbb{N}}
\newcommand{\Z}{\mathbb{Z}}
\newcommand{\Q}{\mathbb{Q}}
\newcommand{\I}{\mathbb{I}}
\newcommand{\R}{\mathbb{R}}
\newcommand{\abs}[1]{|#1|}
\newcommand{\card}[1]{\abs{#1}}
\newcommand{\st}{\mathrel{:}}

\title{MATH 355: Homework 1}
\author{Alexander Lee}

\begin{document}

\maketitle

\begin{exercise}[1.3.8]
  \begin{enumerate}[label={(\alph*)}]
    \item Supremum: $1$. Infimum: $0$.
    \item Supremum: $1$. Infimum: $-1$.
    \item Supremum: $1$. Infimum: $\frac{1}{4}$.
    \item Supremum: $1$. Infimum: $0$.
  \end{enumerate}
\end{exercise}

\begin{exercise}[1.3.9]
  \begin{enumerate}[label={(\alph*)}]
    \item Since $\sup(A) < \sup(B)$, we have $\frac{\sup(B) - \sup(A)}{2} > 0$.
      Set $m = \frac{\sup(B) - \sup(A)}{2}$. By Lemma 1.3.8, we know $\sup(B) -
      m < b$ for some $b \in B$. Because
      \begin{align*}
        \sup(B) - m &= \sup(B) - \frac{\sup(B) - \sup(A)}{2} \\
        &= \frac{2 \sup(B) - \sup(B) + \sup(A)}{2} \\
        &= \frac{\sup(B) + \sup(A)}{2} \\
        &= \frac{\sup(B) - \sup(A)}{2} + \sup(A) \\
        &> \sup(A),
      \end{align*}
      we know that $\sup(B) - m$ is an upper bound for $A$. Since $\sup(B) - m <
      b$, $b$ is also an upper bound for $A$. Observe that $b \in B$, so we are
      done.
    \item Let $A = B = (0, 1)$. We have $\sup(A) = \sup(B) = 1$, but all upper
      bounds for $A$ are not elements of $B$.
  \end{enumerate}
\end{exercise}

\begin{exercise}[1.4.1]
  \begin{enumerate}[label={(\alph*)}]
    \item Given $a, b \in \Q$, we can write $a = \frac{p_1}{q_1}$ and $b =
      \frac{p_2}{q_2}$ for some $p_1, q_1, p_2, q_2 \in \Z$ with $q_1, q_2 \neq
      0$.  Then, $ab = (\frac{p_1}{q_1}) (\frac{p_2}{q_2}) = \frac{p_1 p_2}{q_1
      q_2}$. Since $p_1 p_2, q_1 q_2 \in \Z$ and $q_1 q_2 \neq 0$ because $q_1,
      q_2 \neq 0$, we have that $ab \in \Q$. Next, we can assume that if $a$ or
      $b$ are negative, the numerator is negative and the denominator is
      positive. Then, $a + b = \frac{p_1}{q_1} + \frac{p_2}{q_2} = \frac{p_1 q_2
      + p_2 q_1}{q_1 + q_2}$.  Since $p_1 q_2 + p_2 q_1 \in \Z$ and $q_1 + q_2
      \neq 0$ because $q_1, q_2 > 0$, we have that $a + b \in \Q$.
    \item Given $a \in \Q$, we can write $a = \frac{p}{q}$ for some $p, q \in
      \Z$ with $q \neq 0$. Since $a \neq 0$, we have $p \neq 0$ as well.
      Suppose $t \in \I$. Then, $a + t = \frac{p}{q} + \frac{qt}{q} = \frac{p +
      qt}{q}$. Since $p, q \neq 0$, we have $p + qt \neq 0$. Furthermore,
      because $qt \notin \Z$, we have $p + qt \notin \Z$. Therefore, $a + t
      \notin \Q$, implying that $a + t \in \I$.  Next, consider $at =
      \frac{p}{q} (t) = \frac{pt}{q}$. Since $p \neq 0$, we have that $pt \neq
      0$. Furthermore, because $pt \neq \Z$, we have $at \notin \Q$, which
      implies that $at \in \I$.
    \item Given $s, t \in \I$, we have $s + t, st \in \I$.
  \end{enumerate}
\end{exercise}

\begin{exercise}[1.4.3]
  Suppose towards a contradiction that $\bigcap_{n=1}^{\infty} (0, 1/n) \neq
  \emptyset$. Set $A = \bigcap_{n=1}^{\infty} (0, 1/n)$. For all $n \in \N$,
  $1/n$ is thus an upper bound for $A$. Since $A \neq \emptyset$, there exists
  an $a \in A$ such that $a > 0$. By the Archimedean Property, there exists an
  $n \in \N$ such that $1/n < a$, which is a contradiction since our assumption
  implied that for all $n \in \N$, $1/n$ is an upper bound for A. Therefore,
  $\bigcap_{n=1}^{\infty} (0, 1/n) = A = \emptyset$.
\end{exercise}

\begin{exercise}[1.4.4]
  By the construction of $T$, $b$ is an upper bound for $T$ since $b \ge t$ for
  all $t \in T$. Given an arbitrary upper bound $u \in \R$ for $T$, suppose
  towards a contradiction that $u < b$. By the density of $\Q$ in $\R$, there
  exists an $r \in \Q$ such that $u < r < b$. Since $u < r$ and $r \in T$, $u$
  is therefore not an upper bound for $T$, which contradicts our previous
  assumption that $u$ is an upper bound for $T$. Hence, it must be that $b \le
  u$, implying that $\sup(T) = b$.
\end{exercise}

\begin{exercise}[1.5.4]
  \begin{enumerate}[label={(\alph*)}]
    \item We know that $(-1, 1) \sim \R$ with the function $f(x) = x / (x^2 -
      1)$. Therefore, what is left to show is that $(-1, 1) \sim (a, b)$. Let $f
      : (-1, 1) \rightarrow (a, b)$ be given by $f(x) = \frac{b - a}{2} x +
      \frac{a + b}{2}$. (1--1) Suppose $f(x_1) = f(x_2)$ for some $x_1, x_2
      \in (-1, 1)$. Then,
      \begin{align*}
        f(x_1) = f(x_2) &\implies \frac{b - a}{2} x_1 + \frac{a + b}{2} =
        \frac{b - a}{2} x_2 + \frac{a + b}{2} \\
        &\implies \frac{b - a}{2} x_1 = \frac{b - a}{2} x_2 \\
        &\implies x_1 = x_2.
      \end{align*}
      (Onto) Given $y \in (a, b)$, let $x = \frac{2y - a - b}{b - a}$. Then,
      \begin{align*}
        f(x) &= f\left(\frac{2y - a - b}{b - a}\right) \\
        &= \frac{b - a}{2} \cdot \frac{2y - a - b}{b - a} + \frac{a + b}{2} \\
        &= \frac{2y - a - b}{2} + \frac{a + b}{2} \\
        &= \frac{2y}{2} \\
        &= y.
      \end{align*}
      Since $f$ is 1--1 and onto, we have $(-1, 1) \sim (a, b)$. Thus, $(a, b)
      \sim \R$.
    \item Consider the function $f : (a, \infty) \rightarrow \R$ given by $f(x)
      = \ln(x + a)$. (1--1) Suppose $f(x_1) = f(x_2)$ for some $x_1, x_2 \in (a,
      \infty)$. Then,
      \begin{align*}
        f(x_1) = f(x_2) &\implies \ln(x_1 + a) = \ln(x_2 + a) \\
        &\implies x_1 + a = x_2 + a \\
        &\implies x_1 = x_2.
      \end{align*}
      (Onto) Given $y \in \R$, let $x = e^y - a$. Then,
      \begin{align*}
        f(x) &= f(e^y - a) \\
        &= \ln(e^y - a + a) \\
        &= \ln(e^y) \\
        &= y.
      \end{align*}
      Since $f$ is 1--1 and onto, we have $(a, \infty) \sim \R$.
    \item Let $T = \Q \cap (0, 1) = \{r_1, r_2, r_2, \ldots\}$. Next, consider
      the function $f : \lbrack 0, 1 \rparen \rightarrow (0, 1)$ given by
      \[
        f(x) =
        \begin{cases}
          r_1 &\quad x = 0 \\
          r_{i+1} &\quad x = r_i \\
          x &\quad x \notin T
        \end{cases},
        i \in \N.
      \]
      (1--1) Suppose $f(x_1) = f(x_2) = y$ for some $x_1, x_2 \in \lbrack 0, 1
      \rparen$. If $y = r_1$, then $x_1 = 0 = x_2$. If $y = r_{i+1}$ for some $i
      \in \N$, then $x_1 = r_i = x_2$. If $y = x \notin T$, then $x_1 = x =
      x_2$. (Onto) Suppose $y \in (0, 1)$. If $y = r_1$, then $f(0) = r_1 = y$.
      If $y = r_{i+1}$, then $f(r_i) = r_{i+1} = y$. If $y = x \notin T$, then
      $f(x) = x = y$. Therefore, since $f$ is 1--1 and onto, we have $\lbrack 0,
      1 \rparen \sim (0, 1)$.
  \end{enumerate}
\end{exercise}

\begin{exercise}[1.5.5]
  \begin{enumerate}[label={(\alph*)}]
    \item $A \sim A$ for every set $A$ because the identity function $f : A
      \rightarrow A$ given by $f(x) = x$ is always 1--1 and onto.
    \item Given sets $A$ and $B$, $A \sim B$ is equivalent to asserting $B \sim
      A$ because any 1--1 and onto function $f : A \rightarrow B$ has an inverse
      function $f^{-1}: B \rightarrow A$ that is also 1--1 and onto.
    \item Given three sets $A$, $B$, and $C$, suppose $A \sim B$ and $B \sim C$.
      Thus, there exist 1--1 and onto functions $f : A \rightarrow B$ and $g : B
      \rightarrow C$. Since $g \circ f : A \rightarrow C$ is also 1--1 and onto,
      we have that $A \sim C$.
  \end{enumerate}
\end{exercise}

\begin{exercise}[1.6.5]
  \begin{enumerate}[label={(\alph*)}]
    \item $P(A) = \{\emptyset, \{a\}, \{b\}, \{c\}, \{a, b\}, \{a, c\}, \{b,
      c\}, \{a, b, c\}\}$
    \item We proceed with induction on $n$. Our base case is $n = 0$. That is,
      $A = \emptyset$, which only has one subset, namely $\emptyset$. Since $2^0
      = 1$, the statement is true for the base case. Now, assume that if $A$ is
      finite with $n = k$ elements, then $P(A)$ has $2^k$ elements. We want to
      show that the statement is true if $A$ is finite with $n = k + 1$
      elements. Suppose $A$ is finite with $k + 1$ elements. Then, there exists
      a set $B$ such that $A = B \cup \{a\}$ with $a \notin B$. Notice that
      $\card{B} = k$ and $B \subseteq A$. Further observe that $P(A) = P(B) \cup
      \{x \cup \{a\} \st x \in P(B)\}$. By the inductive hypothesis, $P(B) =
      2^k$. Also, $\card{\{x \cup \{a\} \st x \in P(B)\}} = P(B) = 2^k$.
      Since $P(B)$ and $\{x \cup \{a\} \st x \in P(B)\}$ are disjoint, we have
      $\card{P(A)} = 2^k + 2^k = 2(2^k) = 2^{k+1}$.
  \end{enumerate}
\end{exercise}

\end{document}
