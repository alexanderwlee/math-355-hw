\documentclass{amsart}

\usepackage{enumitem}
\usepackage{mathtools}

\theoremstyle{definition}
\newtheorem{exercise}{Exercise}

\DeclarePairedDelimiter\abs{\lvert}{\rvert} % absolute value
% Swap the definition of \abs*, so that \abs
% resizes the size of the bars, and the starred version does not.
\makeatletter
\let\oldabs\abs%
\def\abs{\@ifstar{\oldabs}{\oldabs*}}
\makeatother

\newcommand{\N}{\mathbb{N}}
\newcommand{\Z}{\mathbb{Z}}
\newcommand{\Q}{\mathbb{Q}}
\newcommand{\I}{\mathbb{I}}
\newcommand{\R}{\mathbb{R}}
\newcommand{\card}[1]{\abs{#1}}
\newcommand{\st}{\mathrel{:}}

\title{MATH 355: Homework 3}
\author{Alexander Lee}

\begin{document}

\maketitle

\begin{exercise}[2.2.1]
  Example: consider the sequence $(a_n)$, where $a_n = {(-1)}^n$. The sequence
  verconges to $1$ if we set $\epsilon = 3$. This sequence is also divergent.
  Since this sequence also verconges to $-1$ if we set $\epsilon = 3$, a
  sequence can verconge to two different values. This strange definition
  describes that a sequence is bounded.
\end{exercise}

\begin{exercise}[2.2.2]
  \begin{enumerate}[label={(\alph*)}]
    \item Let $\epsilon > 0$ be arbitrary. Choose $N \in \N$ such that $N >
      \frac{3}{25 \epsilon} - \frac{4}{5}$. Let $n \ge N$. Then,
      \begin{align*}
        \abs{a_n - \frac{2}{5}} &= \abs{\frac{2n + 1}{5n + 4} - \frac{2}{5}} \\
        &= \abs{\frac{5 (2n + 1) - 2 (5n + 4)}{25n + 20}} \\
        &= \abs{\frac{10n + 5 - 10n - 8}{25n + 20}} \\
        &= \abs{\frac{-3}{25n + 20}} \\
        &= \frac{3}{25n + 20} \\
        &\le \frac{3}{25N + 20} \\
        &< \frac{3}{25 (\frac{3}{25\epsilon} - \frac{4}{5}) + 20} \\
        &= \frac{3}{\frac{3}{\epsilon} - 20 + 20} \\
        &= \epsilon.
      \end{align*}
      Hence, $\abs{a_n - \frac{2}{5}} < \epsilon$.
    \item Let $\epsilon > 0$ be arbitrary. Choose $N \in \N$ such that $N >
      \frac{2}{\epsilon}$. Let $n \ge N$. Then,
      \begin{align*}
        \abs{a_n - 0} &= \abs{\frac{2n^2}{n^3 + 3}} \\
        &= \frac{2n^2}{n^3 + 3} \\
        &< \frac{2n^2}{n^3} \\
        &= \frac{2}{n} \\
        &\le \frac{2}{N} \\
        &< \frac{2}{\frac{2}{\epsilon}} \\
        &= \epsilon.
      \end{align*}
      Hence, $\abs{a_n - 0} < \epsilon$.
    \item Let $\epsilon > 0$ be arbitrary. Choose $N \in \N$ such that $N >
      \frac{1}{\epsilon^3}$. Let $n \ge N$. Then,
      \begin{align*}
        \abs{a_n - 0} &= \abs{\frac{\sin(n^2)}{\sqrt[3]{n}}} \\
        &\le \frac{1}{\sqrt[3]{n}} \\
        &\le \frac{1}{\sqrt[3]{N}} \\
        &< \frac{1}{\sqrt[3]{\frac{1}{\epsilon^3}}} \\
        &= \epsilon.
      \end{align*}
      Hence, $\abs{a_n - 0} < \epsilon$.
  \end{enumerate}
\end{exercise}

\begin{exercise}[2.2.4]
  \begin{enumerate}[label={(\alph*)}]
    \item Consider the sequence $(a_n)$, where $a_n = {(-1)}^n$. $(a_n)$ has an
      infinite number of ones, but does not converge to one since it diverges.
    \item TODO
    \item TODO
  \end{enumerate}
\end{exercise}

\begin{exercise}[2.2.5]
  \begin{enumerate}[label={(\alph*)}]
    \item Let $a_n = [[5 / n]]$. We claim that $\lim{a_n} = 0$. Let $\epsilon >
      0$ be arbitrary. Choose $N \in \N$ such that $N > 5/\epsilon$. Let $n \ge
      N$. Then,
      \begin{align*}
        \abs{a_n - 0} &= \abs{[[5 / n]]} \\
        &= [[5 / n]] \\
        &\le 5 / n \\
        &\le 5 / N \\
        &< 5 / (5 / \epsilon) \\
        &= \epsilon.
      \end{align*}
      Hence, $\abs{a_n - 0} < \epsilon$.
    \item Let $a_n = [[(12 + 4n) / 3n]]$. We claim that $\lim{a_n} = 1$. Let
      $\epsilon > 0$ be arbitrary. Choose $N \in \N$ such that $N >
      \frac{4}{\epsilon - \frac{1}{3}}$. Let $n \ge N$. Then,
      \begin{align*}
        \abs{a_n - 1} &= \abs{[[(12 + 4n) / 3n]] - 1} \\
        &= \abs{[[(12 + 4n) / 3n - 1]]} \\
        &= \abs{[[(12 + 4n - 3n) / 3n]]} \\
        &= \abs{[[(12 + n) / 3n]]} \\
        &= [[(12 + n) / 3n]] \\
        &\le (12 + n) / 3n \\
        &= 4 / n + 1 / 3 \\
        &\le 4 / N + 1 / 3 \\
        &< 4 / (4 / (\epsilon - 1 / 3)) + 1 / 3 \\
        &= \epsilon - 1 / 3 + 1 / 3 \\
        &= \epsilon.
      \end{align*}
      Hence, $\abs{a_n - 1} < \epsilon$.
  \end{enumerate}
\end{exercise}

\begin{exercise}[2.2.7]
  \begin{enumerate}[label={(\alph*)}]
    \item The sequence ${(-1)}^n$ is frequently in the set $\{1\}$.
    \item The definition of eventually is stronger than that of frequently,
      since eventually implies frequently.
    \item A sequence $(a_n)$ converges to $a$ if, given any
      $\epsilon$-neighborhood $V_\epsilon(a)$ of $a$, $(a_n)$ is eventually in
      the set $V_\epsilon(a)$. Eventually is the term we want.
    \item $(x_n)$ is not necessarily eventually in the interval $(1.9, 2.1)$.
      For instance, consider the sequence $(1, 2, 1, 2, \ldots)$. However,
      $(x_n)$ is frequently in $(1.9, 2.1)$.
  \end{enumerate}
\end{exercise}

\end{document}
