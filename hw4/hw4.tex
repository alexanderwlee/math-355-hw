\documentclass{amsart}

\usepackage{enumitem}
\usepackage{mathtools}

\theoremstyle{definition}
\newtheorem{exercise}{Exercise}

\DeclarePairedDelimiter\abs{\lvert}{\rvert} % absolute value
% Swap the definition of \abs*, so that \abs
% resizes the size of the bars, and the starred version does not.
\makeatletter
\let\oldabs\abs%
\def\abs{\@ifstar{\oldabs}{\oldabs*}}
\makeatother

\newcommand{\N}{\mathbb{N}}
\newcommand{\Z}{\mathbb{Z}}
\newcommand{\Q}{\mathbb{Q}}
\newcommand{\I}{\mathbb{I}}
\newcommand{\R}{\mathbb{R}}
\newcommand{\card}[1]{\abs{#1}}
\newcommand{\st}{\mathrel{:}}

\title{MATH 355: Homework 4}
\author{Alexander Lee}

\begin{document}

\maketitle

\begin{exercise}[2.3.3]
  Let $\epsilon > 0$ be arbitrary. Since $\lim{x_n} = l$, there exists an $N_1
  \in \N$ such that for all $n \ge N_1$, we have that $\abs{x_n - l} <
  \epsilon$. Thus, we have $l - \epsilon < x_n < l + \epsilon$. Similarly, since
  $\lim{z_n} = l$, there exists an $N_2 \in \N$ such that for all $n \ge N_2$,
  we have that $\abs{z_n - l} < \epsilon$. Hence, we also have $l - \epsilon <
  z_n < l + \epsilon$. Let $N = \max\{N_1, N_2\}$. Because we are given that
  $x_n \le y_n \le z_n$, it follows that for all $n \ge N$, we have $l -
  \epsilon < x_n \le y_n \le z_n < l + \epsilon$, which implies that $l -
  \epsilon < y_n < l + \epsilon$ and $\abs{y_n - l} < \epsilon$. Therefore,
  $\lim{y_n} = l$ as well.
\end{exercise}

\begin{exercise}[2.3.4]
  \begin{enumerate}[label={(\alph*)}]
    \item
      \begin{align*}
        \lim\left(\frac{1 + 2a_n}{1 + 3a_n - 4{a_n}^2}\right) &= \frac{\lim{1} +
        2 \lim{a_n}}{\lim{1} + 3 \lim{a_n} - 4 \lim{a_n} \lim{a_n}} \\
        &= \frac{1 + 2 \cdot 0}{ 1 + 3 \cdot 0 - 4 \cdot 0 \cdot 0} \\
        &= 1
      \end{align*}
    \item
      \begin{align*}
        \lim\left(\frac{{(a_n + 2)}^2 - 4}{a_n}\right) &= \frac{\lim(a_n + 2)
        \lim(a_n + 2) - \lim{4}}{\lim{a_n}} \\
        &= \frac{(\lim{a_n} + \lim{2}) (\lim{a_n} + \lim{2}) - 4}{\lim{a_n}} \\
        &= \frac{(\lim{a_n} + 2) (\lim{a_n} + 2) - 4}{\lim{a_n}} \\
        &= \frac{{(\lim{a_n})}^2 + 4 \lim{a_n} + 4 - 4}{\lim{a_n}} \\
        &= \frac{\lim{a_n} (\lim{a_n} + 4)}{\lim{a_n}} \\
        &= \lim{a_n} + 4 \\
        &= 0 + 4 \\
        &= 4
      \end{align*}
    \item
      \begin{align*}
        \lim\left(\frac{\frac{2}{a_n} + 3}{\frac{1}{a_n} + 5}\right) &=
        \frac{\frac{\lim{2}}{\lim{a_n}} + \lim{3}}{\frac{\lim{1}}{\lim{a_n}} +
        \lim{5}} \\
        &= \frac{\frac{2}{\lim{a_n}} + 3}{\frac{1}{\lim{a_n}} + 5} \\
        &= \frac{\frac{2 + 3 \lim{a_n}}{\lim{a_n}}}{\frac{1 + 5
        \lim{a_n}}{\lim{a_n}}} \\
        &= \frac{2 + 3 \lim{a_n}}{1 + 5 \lim{a_n}} \\
        &= \frac{2 + 3 \cdot 0}{1 + 5 \cdot 0} \\
        &= \frac{2}{1} \\
        &= 2
      \end{align*}
  \end{enumerate}
\end{exercise}

\begin{exercise}[2.3.7]
  \begin{enumerate}[label={(\alph*)}]
    \item Possible. Consider the divergent sequences $(x_n) = (-1, 1, -1, 1,
      \ldots)$ and $(y_n) = (1, -1, 1, -1, \ldots)$. Then, we have $(x_n + y_n)
      = (0, 0, 0, 0, \ldots)$, which converges to 0.
    \item Impossible. Observe that $y_n = (x_n + y_n) - x_n$. Since $(x_n +
      y_n)$ and $(x_n)$ converge, $(y_n)$ must converge as well by the Algebraic
      Limit Theorem (ii).
    \item Impossible. Since $(1)$ and $(b_n)$ converge, $(1 / b_n)$ must
      converge as well by the Algebraic Limit Theorem (iv). TODO
    \item Impossible. Since $(a_n - b_n)$ is bounded, there exists a number
      $M_1 > 0$ such that $\abs{a_n - b_n} \le M_1$ for all $n \in \N$.
      Furthermore, since $(b_n)$ converges, it is also bounded. Thus, there
      exists a number $M_2 > 0$ such that $\abs{b_n} \le M_2$ for all $n \in
      \N$. As such, $\abs{a_n} \le \abs{a_n - b_n} + \abs{b_n} \le M_1 + M_2$,
      so $(a_n)$ must be bounded too.
    \item Impossible. TODO
  \end{enumerate}
\end{exercise}

\begin{exercise}[2.3.10]
  \begin{enumerate}[label={(\alph*)}]
    \item False. Consider $(a_n) = (b_n) = (1, 2, 3, 4, \ldots)$. Since $(a_n -
      b_n) = (0, 0, 0, 0, \ldots)$, we have that $\lim(a_n - b_n) = 0$. However,
      since $(a_n)$ and $(b_n)$ are not bounded, they do not even converge to a
      limit.
    \item True. Since $(b_n) \rightarrow b$, for any $\epsilon > 0$, there
      exists an $N \in \N$ such that for all $n \ge N$, we have that $\abs{b_n -
      b} < \epsilon$. Since $\abs{\abs{b_n} - \abs{b}} \le \abs{b_n - b}$, we
      also have that $\abs{\abs{b_n} - \abs{b}} < \epsilon$. Thus, $\abs{b_n}
      \rightarrow \abs{b}$.
    \item True. Note that $b_n = (b_n - a_n) + a_n$. Thus, by the Algebraic
      Limit Theorem (ii), we have $\lim{b_n} = \lim(b_n - a_n) + \lim{a_n} = 0 +
      a = a$.
    \item True. Since $(a_n) \rightarrow 0$, for any $\epsilon > 0$, there
      exists an $N \in \N$ such that for all $n \ge N$, we have that $\abs{a_n}
      < \epsilon$, implying that $-\epsilon < a_n < \epsilon$. Since we have
      that $\abs{b_n - b} \le a_n$, we therefore have $\abs{b_n - b} \le a_n <
      \epsilon$. Hence, $(b_n) \rightarrow b$.
  \end{enumerate}
\end{exercise}

\begin{exercise}[2.3.12]
  \begin{enumerate}[label={(\alph*)}]
    \item True. Given $b \in B$, we know that $a_n \ge b$ for all $n \in \N$
      since $a_n$ is an upper bound for $B$. By the Order Limit Theorem (iii),
      we have that $a = \lim{a_n} \ge b$. Thus, $a$ is also an upper bound for
      $B$.
    \item True. Suppose that every $a_n$ is in the complement of the interval
      $(0,1)$. Next, suppose towards a contradiction that $a$ is in $(0,1)$.
      Since $(a_n) \rightarrow a$, there exists an $\epsilon$-neighborhood
      $V_\epsilon(a) \subseteq (0,1)$ that contains all but finitely many terms
      of $(a_n)$. That is, all but finitely many terms of $(a_n)$ are in
      $(0,1)$. This is a contradiction since we previously assumed that every
      $a_n$ is not in $(0,1)$.
    \item False. Consider the sequence $(a_n) = (1.4, 1.41, 1.414, 1.4142,
      1.41421, \ldots)$, where every consequent term is a decimal approximation
      of $\sqrt{2}$ that specifies an additional decimal. Each term in the
      sequence $(a_n)$ is rational, however, by construction, $(a_n) \rightarrow
      \sqrt{2}$, which is irrational.
  \end{enumerate}
\end{exercise}

\begin{exercise}[2.4.1]
  \begin{enumerate}[label={(\alph*)}]
    \item We first show by induction that $(x_n)$ is decreasing (i.e., $x_n \ge
      x_{n+1}$). Since $x_1 = 3$ and $x_2 = \frac{1}{4 - 3} = 1$, we have that
      $x_n \ge x_{n+1}$. Next, suppose that $3 = x_1 \ge x_2 \ge \cdots \ge x_n
      \ge x_{n+1}$. We want to show that $x_{n+1} \ge x_{n+2}$. Consider
      \begin{align*}
        x_{n+1} - x_{n+2} &= \frac{1}{4 - x_n} - \frac{1}{4 - x_{n+1}} \\
        &= \frac{4 - x_{n+1} - (4 - x_n)}{(4 - x_n) (4 - x_{n+1})} \\
        &= \frac{x_n - x_{n+1}}{(4 - x_n) (4 - x_{n+1})}.
      \end{align*}
      Since $x_n \ge x_{n+1}$ by our inductive hypothesis, we know that $x_n -
      x_{n+1} \ge 0$. Furthermore, by our inductive hypothesis, we also know
      that $3 \ge x_n \ge x_{n+1} \implies -3 \le -x_n \le -x_{n+1} \implies 1 =
      4 - 3 \le 4 - x_n \le 4 - x_{n+1}$. Hence, we have that $x_{n+1} - x_{n+2}
      \implies x_{n+1} \ge x_{n+2}$. Therefore, $(x_n)$ is decreasing. Next, we
      show that $(x_n)$ is bounded. Since $x_n < 4$ for all $n \in \N$, we have
      that $x_{n+1} = \frac{1}{4 - x_n}> 0$. Thus, $(x_n)$ is bounded. As such,
      $(x_n)$ converges by the Monotone Convergence Theorem.
    \item Suppose $\lim{x_n} = l$. By definition, given any $\epsilon > 0$,
      there exists an $N \in \N$ such that for all $n \ge N$, we have $\abs{x_n
      - l} < \epsilon$. Since $x_{n+1} \le x_n$ for all $n \in \N$, we have
      $\abs{x_{n+1} - l} \le \abs{x_n - l} < \epsilon$. Thus, $\lim{x_{n+1}} = l
      = \lim_{x_n}$.
    \item Suppose $\lim{x_n} = \lim{x_{n+1}} = l$. Then,
      \begin{align*}
        \lim{x_{n+1}} = \lim\left(\frac{1}{4 - x_n}\right) &\implies l =
        \frac{1}{4 - l} \\
        &\implies 4l - l^2 = 1 \\
        &\implies 0 = l^2 - 4l + 1 \\
        &\implies 0 = {(l - 2)}^2 - 3 \\
        &\implies 3 = {(l - 2)}^2 \\
        &\implies \pm \sqrt{3} = l - 2 \\
        &\implies l = 2 \pm \sqrt{3}.
      \end{align*}
      By the Order Limit Theorem (iii), we know that $\lim{x_n} = l \le x_n = 3$.
      Thus, $\lim{x_n} = 2 - \sqrt{3}$.
  \end{enumerate}
\end{exercise}

\begin{exercise}[2.4.8]
  \begin{enumerate}[label={(\alph*)}]
    \item The sequence of partial sums is defined by
      \begin{align*}
        s_m &= \frac{1}{2} + \frac{1}{2^2} + \frac{1}{2^3} + \cdots +
        \frac{1}{2^m} \\
        &= \frac{2^{m-1} + 2^{m-2} + \cdots + 1}{2^m} \\
        &= \frac{2^m - 1}{2^m} \\
        &= 1 - \frac{1}{2^m}.
      \end{align*}
      Since the sequence is increasing and bounded above
      by $1$, the sequence of partial sums converges by the Monotone Convergence
      Theorem. Thus, the series converges.
    \item The sequence of partial sums is defined by
      \begin{align*}
        s_m &= \frac{1}{1 \cdot 2} + \frac{1}{2 \cdot 3} + \frac{1}{3 \cdot 4} +
        \cdots + \frac{1}{n (n + 1)} \\
        &= \left(1 - \frac{1}{2}\right) + \left(\frac{1}{2} - \frac{1}{3}\right)
        + \cdots + \left(\frac{1}{n} - \frac{1}{n + 1}\right) \\
        &= 1 - \frac{1}{n + 1}.
      \end{align*}
      Since the sequence of partial sums is increasing and bounded above by $1$,
      the it converges by the Monotone Convergence Theorem. Thus, the series
      converges.
    \item The sequence of partial sums is defined by
      \begin{align*}
        s_m &= \log\left(\frac{1}{2}\right) + \log\left(\frac{2}{3}\right) +
        \log\left(\frac{3}{4}\right) + \cdots + \log\left(\frac{n+1}{n}\right)
        \\
        &= \log\left(\frac{1 \times 2 \times 3 \times \cdots \times n + 1}{2
        \times 3 \times 4 \times \cdots \times n}\right) \\
        &= \log\left(\frac{n + 1}{1}\right) \\
        &= \log(n+1).
      \end{align*}
      Since the sequence of partial sums is unbounded, it diverges.
  \end{enumerate}
\end{exercise}

\begin{exercise}[2.4.9]
  TODO
\end{exercise}

\end{document}
