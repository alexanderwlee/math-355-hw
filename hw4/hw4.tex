\documentclass{amsart}

\usepackage{enumitem}
\usepackage{mathtools}

\theoremstyle{definition}
\newtheorem{exercise}{Exercise}

\DeclarePairedDelimiter\abs{\lvert}{\rvert} % absolute value
% Swap the definition of \abs*, so that \abs
% resizes the size of the bars, and the starred version does not.
\makeatletter
\let\oldabs\abs%
\def\abs{\@ifstar{\oldabs}{\oldabs*}}
\makeatother

\newcommand{\N}{\mathbb{N}}
\newcommand{\Z}{\mathbb{Z}}
\newcommand{\Q}{\mathbb{Q}}
\newcommand{\I}{\mathbb{I}}
\newcommand{\R}{\mathbb{R}}
\newcommand{\card}[1]{\abs{#1}}
\newcommand{\st}{\mathrel{:}}

\title{MATH 355: Homework 4}
\author{Alexander Lee}

\begin{document}

\maketitle

\begin{exercise}[2.3.3]
  Given $x_n \le y_n \le z_n$ for all $n \in \N$ and $\lim{x_n} = \lim{z_n} =
  l$, we know from the Order Limit Theorem (ii) that $l \le \lim{y_n} \le l$
  for all $n \in \N$. Thus, we have that $\lim{y_n} = l$ as well.
\end{exercise}

\begin{exercise}[2.3.4]
  \begin{enumerate}[label={(\alph*)}]
    \item
      \begin{align*}
        \lim\left(\frac{1 + 2a_n}{1 + 3a_n - 4{a_n}^2}\right) &= \frac{\lim{1} +
        2 \lim{a_n}}{\lim{1} + 3 \lim{a_n} - 4 \lim{a_n} \lim{a_n}} \\
        &= \frac{1 + 2 \cdot 0}{ 1 + 3 \cdot 0 - 4 \cdot 0 \cdot 0} \\
        &= 1
      \end{align*}
    \item
      \begin{align*}
        \lim\left(\frac{{(a_n + 2)}^2 - 4}{a_n}\right) &= \frac{\lim(a_n + 2)
        \lim(a_n + 2) - \lim{4}}{\lim{a_n}} \\
        &= \frac{(\lim{a_n} + \lim{2}) (\lim{a_n} + \lim{2}) - 4}{\lim{a_n}} \\
        &= \frac{(\lim{a_n} + 2) (\lim{a_n} + 2) - 4}{\lim{a_n}} \\
        &= \frac{{(\lim{a_n})}^2 + 4 \lim{a_n} + 4 - 4}{\lim{a_n}} \\
        &= \frac{\lim{a_n} (\lim{a_n} + 4)}{\lim{a_n}} \\
        &= \lim{a_n} + 4 \\
        &= 0 + 4 \\
        &= 4
      \end{align*}
    \item
      \begin{align*}
        \lim\left(\frac{\frac{2}{a_n} + 3}{\frac{1}{a_n} + 5}\right) &=
        \frac{\frac{\lim{2}}{\lim{a_n}} + \lim{3}}{\frac{\lim{1}}{\lim{a_n}} +
        \lim{5}} \\
        &= \frac{\frac{2}{\lim{a_n}} + 3}{\frac{1}{\lim{a_n}} + 5} \\
        &= \frac{\frac{2 + 3 \lim{a_n}}{\lim{a_n}}}{\frac{1 + 5
        \lim{a_n}}{\lim{a_n}}} \\
        &= \frac{2 + 3 \lim{a_n}}{1 + 5 \lim{a_n}} \\
        &= \frac{2 + 3 \cdot 0}{1 + 5 \cdot 0} \\
        &= \frac{2}{1} \\
        &= 1
      \end{align*}
  \end{enumerate}
\end{exercise}

\begin{exercise}[2.3.7]
  \begin{enumerate}[label={(\alph*)}]
    \item Consider the divergent sequences $(x_n) = (-1, 1, -1, 1, \ldots)$ and
      $(y_n) = (1, -1, 1, -1, \ldots)$. Then, we have $(x_n + y_n) = (0, 0, 0,
      0, \ldots)$, which converges to 0.
    \item TODO
    \item Impossible, by the Algebraic Limit Theorem (iv).
    \item TODO
    \item TODO
  \end{enumerate}
\end{exercise}

\begin{exercise}[2.3.10]
  \begin{enumerate}[label={(\alph*)}]
    \item Given $\lim(a_n - b_n) = 0$, we have that $\lim{a_n} - \lim{b_n} =
      0$ by the Algebraic Limit Theorem (ii). Thus, we also have that $\lim{a_n}
      = \lim{b_n}$.
    \item TODO
    \item Given $(a_n) \rightarrow a$ and $(b_n - a_n) \rightarrow 0$, we have
      that $\lim{a_n} = a$ and $\lim(b_n - a_n) = 0$. The later limit implies
      that $\lim{b_n} = \lim{a_n}$ by (a). Thus, we have that $\lim{b_n} =
      \lim{a_n} = a$ and so $(b_n) \rightarrow a$.
    \item TODO
  \end{enumerate}
\end{exercise}

\begin{exercise}[2.3.12]
  \begin{enumerate}[label={(\alph*)}]
    \item True
      % By the Order Limit Theorem (iii).
    \item True.
      % Proof by contradiction. Assume that a is in (0,1). Since (a_n) -> a, all
      % but finitely many terms of (a_n) are in (0,1). Contradiction since every
      % a_n is not in (0,1).
    \item TODO
  \end{enumerate}
\end{exercise}

\begin{exercise}[2.4.1]
  \begin{enumerate}[label={(\alph*)}]
    \item We first show by induction that $(x_n)$ is decreasing (i.e., $x_n \ge
      x_{n+1}$). Since $x_1 = 3$ and $x_2 = \frac{1}{4 - 3} = 1$, we have that
      $x_n \ge x_{n+1}$. Next, suppose that $3 = x_1 \ge x_2 \ge \cdots \ge x_n
      \ge x_{n+1}$. We want to show that $x_{n+1} \ge x_{n+2}$. Consider
      \begin{align*}
        x_{n+1} - x_{n+2} &= \frac{1}{4 - x_n} - \frac{1}{4 - x_{n+1}} \\
        &= \frac{4 - x_{n+1} - (4 - x_n)}{(4 - x_n) (4 - x_{n+1})} \\
        &= \frac{x_n - x_{n+1}}{(4 - x_n) (4 - x_{n+1})}.
      \end{align*}
      Since $x_n \ge x_{n+1}$ by our inductive hypothesis, we know that $x_n -
      x_{n+1} \ge 0$. Furthermore, by our inductive hypothesis, we also know
      that $3 \ge x_n \ge x_{n+1} \implies -3 \le -x_n \le -x_{n+1} \implies 1 =
      4 - 3 \le 4 - x_n \le 4 - x_{n+1}$. Hence, we have that $x_{n+1} - x_{n+2}
      \implies x_{n+1} \ge x_{n+2}$. Therefore, $(x_n)$ is decreasing. Next, we
      show that $(x_n)$ is bounded. Since $x_n < 4$ for all $n \in \N$, we have
      that $x_{n+1} = \frac{1}{4 - x_n}> 0$. Thus, $(x_n)$ is bounded. As such,
      $(x_n)$ converges by the Monotone Convergence Theorem.
    \item Suppose $\lim{x_n} = l$. By definition, given any $\epsilon > 0$,
      there exists an $N \in \N$ such that for all $n \ge N$, we have $\abs{x_n
      - l} < \epsilon$. Since $x_{n+1} \le x_n$ for all $n \in \N$, we have
      $\abs{x_{n+1} - l} \le \abs{x_n - l} < \epsilon$. Thus, $\lim{x_{n+1}} = l
      = \lim_{x_n}$.
    \item Suppose $\lim{x_n} = \lim{x_{n+1}} = l$. Then,
      \begin{align*}
        \lim{x_{n+1}} = \lim\left(\frac{1}{4 - x_n}\right) &\implies l =
        \frac{1}{4 - l} \\
        &\implies 4l - l^2 = 1 \\
        &\implies 0 = l^2 - 4l + 1 \\
        &\implies 0 = {(l - 2)}^2 - 3 \\
        &\implies 3 = {(l - 2)}^2 \\
        &\implies \pm \sqrt{3} = l - 2 \\
        &\implies l = 2 \pm \sqrt{3}.
      \end{align*}
      By the Order Limit Theorem (iii), we know that $\lim{x_n} = l \le x_n = 3$.
      Thus, $\lim{x_n} = 2 - \sqrt{3}$.
  \end{enumerate}
\end{exercise}

\begin{exercise}[2.4.8]
\end{exercise}

\begin{exercise}[2.4.9]
\end{exercise}

\end{document}
