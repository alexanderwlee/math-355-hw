\documentclass{amsart}

\usepackage{enumitem}
\usepackage{mathtools}

\theoremstyle{definition}
\newtheorem{exercise}{Exercise}

\DeclarePairedDelimiter\abs{\lvert}{\rvert} % absolute value
% Swap the definition of \abs*, so that \abs
% resizes the size of the bars, and the starred version does not.
\makeatletter
\let\oldabs\abs%
\def\abs{\@ifstar{\oldabs}{\oldabs*}}
\makeatother

\newcommand{\N}{\mathbb{N}}
\newcommand{\Z}{\mathbb{Z}}
\newcommand{\Q}{\mathbb{Q}}
\newcommand{\I}{\mathbb{I}}
\newcommand{\R}{\mathbb{R}}
\newcommand{\card}[1]{\abs{#1}}
\newcommand{\st}{\mathrel{:}}

\title{MATH 355: Homework 5}
\author{Alexander Lee}

\begin{document}

\maketitle

\begin{exercise}[2.5.1]
  \begin{enumerate}[label={(\alph*)}]
    \item Impossible. Let $(a_n)$ be a sequence with a subsequence $(a_{n_k})$,
      where $(a_{n_k})$ is bounded. By the Balzano-Weierstrass Theorem,
      $(a_{n_k})$ contains a convergent subsequence. Note that this convergent
      subsequence of $(a_{n_k})$ is also a subsequence of $(a_n)$. Thus, $(a_n)$
      contains a convergent subsequence.
    \item Consider the sequences $a_n = \frac{1}{n}$ and $b_n = 1 +
      \frac{1}{n}$. Now, define the sequence $(c_n) = (a_1, b_1, a_2, b_2, a_3,
      b_3, \ldots)$. Notice that $(a_n)$ and $(b_n)$ are both subsequences of
      $(c)n$. Further notice that $(c_n)$ does not contain 0 or 1 as terms but
      $(a_n) \rightarrow 0$ and $(b_n) \rightarrow 1$.
    \item Consider the following sequences.
      \begin{align*}
        (a^1_n) &= (1 + 1/1, 1 + 1/2, 1 + 1/3, \ldots) \\
        (a^2_n) &= (1/2 + 1/1, 1/2 + 1/2, 1/2 + 1/3, \ldots) \\
        (a^3_n) &= (1/3 + 1/1, 1/3 + 1/2, 1/3 + 1/3, \ldots) \\
        &\vdotswithin{\dots}
      \end{align*}
      Notice that $(a^1_n) \rightarrow 1$, $(a^2_n) \rightarrow 1/2$, $(a^3_n)
      \rightarrow 1/3$, and so on. Now consider the sequence $(a_n)$ constructed
      with the top right to bottom left diagonals across the sequences above.
      That is, $(a_n) = (1 + 1/1, 1 + 1/2, 1/2 + 1/1, 1 + 1/3, 1/2 + 1/2, 1/3 +
      1/1, \ldots)$. By construction, $(a_n)$ contains subsequences converging
      to every point in the infinite set $\{1, 1/2, 1/3, \ldots\}$.
    \item TODO
  \end{enumerate}
\end{exercise}

\begin{exercise}[2.5.2]
  \begin{enumerate}[label={(\alph*)}]
    \item True. Since $(x_2, x_3, x_4, \ldots)$ converges and is a proper
      subsequence of $(x_n)$, $(x_n)$ also converges.
    \item True. Suppose towards a contradiction that $(x_n)$ converges. Thus, we
      know that subsequences of $(x_n)$ are convergent, which contradicts the
      assumption that $(x_n)$ contains a divergent subsequence. Therefore,
      $(x_n)$ diverges.
    \item TODO
    \item TODO
  \end{enumerate}
\end{exercise}

\begin{exercise}[2.6.2]
  \begin{enumerate}[label={(\alph*)}]
    \item Consider the sequence $a_n = {(-1)}^n \frac{1}{n}$. $(a_n)$ clearly
      converges to 0, so it is a Cauchy sequence by the Cauchy Criterion. Notice
      that $(a_n)$ is not monotone as well.
    \item Impossible. Suppose $(a_n)$ is a Cauchy sequence. Thus, $(a_n)$ is
      bounded by Lemma 2.6.3. As such, every subsequence of $(a_n)$ must be
      bounded as well.
    \item TODO
    \item Consider the sequence $(a_n) = (0, 1, 0, 2, 0, 3, \ldots)$. Clearly,
      $(a_n)$ is unbounded. However, $(0, 0, 0, \ldots)$ is a subsequence of
      $(a_n)$ that is Cauchy.
  \end{enumerate}
\end{exercise}

\begin{exercise}[2.6.4]
  \begin{enumerate}[label={(\alph*)}]
    \item Let $\epsilon > 0$ be arbitrary. Since $(a_n)$ is a Cauchy sequence,
      there exists an $N_1 \in \N$ such that whenever $m, n \ge N_1$, it follows
      that $\abs{a_n - a_m} < \epsilon / 2$. Similarly, since $(b_n)$ is a
      Cauchy sequence, there exists an $N_2 \in \N$ such that whenever $m, n \ge
      N_2$, it follows that $\abs{b_n - b_m} < \epsilon / 2$. Let $N =
      \max\{N_1, N_2\}$ and suppose $m, n \ge N$. Then,
      \begin{align*}
        \abs{c_n - c_m} &= \abs{\abs{a_n - b_n} - \abs{a_m - b_m}} \\
        &\le \abs{\abs{(a_n - b_n) - (a_m - b_m)}}\ \text{(by the Triangle
        Inequality)} \\
        &= \abs{(a_n - a_m) + (b_m - b_n)} \\
        &\le \abs{a_n - a_m} + \abs{b_m - b_n}\ \text{(by the Triangle
        Inequality)} \\
        &< \epsilon / 2 + \epsilon / 2 \\
        &= \epsilon.
      \end{align*}
      Therefore, $(c_n)$ is a Cauchy sequence.
    \item Let $a_n = 1$. $(a_n)$ is a Cauchy sequence, but $c_n = {(-1)}^n$ is
      divergent and thus not a Cauchy sequence.
    \item Let $a_n = {(-1)}^n \frac{1}{n+1}$. Then,
      \[
        c_n =
        \begin{cases}
          0 &\text{if $n$ is even} \\
          -1 &\text{if $n$ is odd}
        \end{cases},
      \]
      which is divergent and thus not a Cauchy sequence.
  \end{enumerate}
\end{exercise}

\begin{exercise}[2.7.1]
  \begin{enumerate}[label={(\alph*)}]
    \item Let $\epsilon > 0$ be arbitrary. Since $(a_n) \rightarrow 0$, there
      exists an $N \in \N$ such that whenever $m \ge N$, it follows that
      $\abs{a_m} < \epsilon$. Then,
      \begin{align*}
        \abs{s_n - s_m} &= \abs{a_{m+1} - a_{m+2} - \cdots \pm a_n} \\
        &\le \abs{a_{m+1}}\ \text{(since $(a_n)$ is decreasing)} \\
        &\le \abs{a_m} \\
        &< \epsilon
      \end{align*}
      whenever $n > m \ge N$.
  \end{enumerate}
\end{exercise}

\begin{exercise}[2.7.2]
  \begin{enumerate}[label={(\alph*)}]
    \item Converges. $\sum_{n=1}^\infty \frac{1}{2^n}$ converges because it is
      geometric. Also, $0 \le \frac{1}{2^n + n} \le \frac{1}{2^n}$ for all $n
      \in \N$, so by the Comparison Test, $\sum_{n=1}^\infty \frac{1}{2^n + n}$
      converges.
    \item Converges. We know that $\sum_{n=1}^\infty \frac{1}{n^2}$ converges.
      Furthermore, since $0 \le \frac{\abs{\sin(n)}}{n^2} \le \frac{1}{n^2}$ for
      all $n \in \N$, we have that $\sum_{n=1}^\infty \frac{\abs{\sin(n)}}{n^2}$
      converges by the Comparison Test. As such, $\sum_{n=1}^\infty
      \frac{\sin(n)}{n^2}$ converges as well by the Absolute Convergence Test.
    \item Diverges. The series can be expressed as $\sum_{n=1}^\infty
      {(-1)}^{n+1} \frac{n+1}{2n}$. Because $\lim \abs{{(-1)}^{n+1}
      \frac{n+1}{2n}} = \lim \frac{n+1}{2n} \neq 0 = \abs{0}$, we have that
      $\lim {(-1)}^{n+1} \frac{n+1}{2n} \neq 0$ by the contrapositive of
      Exercise 2.3.10 (b). As such, by Theorem 2.7.3, $\sum_{n=1}^\infty
      {(-1)}^{n+1} \frac{n+1}{2n}$ diverges.
    \item TODO
    \item TODO
  \end{enumerate}
\end{exercise}

\begin{exercise}[2.7.4]
  \begin{enumerate}[label={(\alph*)}]
    \item Consider $x_n = \frac{1}{n}$ and $y_n = {(-1)}^n$. Since $\sum x_n$ is
      the harmonic series, it diverges. Since $(y_n) \not\rightarrow 0$, $\sum
      y_k$ diverges. However, $\sum x_n y_n = \sum {(-1)}^n \frac{1}{n}$
      converges by the Alternating Series Test.
    \item Consider $x_n = {(-1)}^n \frac{1}{n}$ and $y_n = {(-1)}^n$. $\sum x_n$
      converges by the Alternating Series Test. Clearly, $(y_n)$ is bounded.
      However, $\sum x_n y_n = \sum \frac{1}{n}$ diverges since it is the
      harmonic series.
    \item Impossible. Since $\sum x_n$ and $\sum (x_n + y_n)$ both converge, we
      have that $\sum y_n = \sum (x_n + y_n) - x_n$ converges as well by the
      Algebraic Limit Theorem for Series.
    \item TODO
  \end{enumerate}
\end{exercise}

\begin{exercise}[2.7.5]
  By the Cauchy Condensation Test, the series $\sum_{n=1}^\infty \frac{1}{n^p}$
  converges if and only if $\sum_{n=0}^\infty 2^n \frac{1}{{(2^n)}^p}$
  converges. Notice that $2^n \frac{1}{{(2^n)}^p} = {(\frac{2}{2^p})}^n$. Thus,
  $\sum_{n=0}^\infty 2^n \frac{1}{{(2^n)}^p}$ converges if and only if
  $\abs{\frac{2}{2^p}} < 1$ by the Geometric Series Test. It follows that
  \begin{align*}
    \abs{\frac{2}{2^p}} < 1 &\iff \frac{2}{2^p} < 1 \\
    &\iff 2 < 2^p \\
    &\iff 1 < 2^{p-1} \\
    &\iff \log_2(1) < p - 1 \\
    &\iff 0 < p - 1 \\
    &\iff 1 < p.
  \end{align*}
  Thus, we must have that $p > 1$.
\end{exercise}

\begin{exercise}[2.7.9]
  \begin{enumerate}[label={(\alph*)}]
    \item TODO
    \item TODO
    \item TODO
  \end{enumerate}
\end{exercise}

\begin{exercise}[3.2.3]
  \begin{enumerate}[label={(\alph*)}]
    \item $\Q$ is not open. For example, no $V_\epsilon(0)$ is contained in $\Q$
      since we can always find an irrational number in $V_\epsilon(0)$. $\Q$ is
      not closed. For example, $\sqrt{2}$ is a limit point of $\Q$ but $\sqrt{2}
      \notin \Q$.
    \item $\N$ is not open. For example, no $V_\epsilon(1)$ is contained in $\N$
      since we can always find a non-natural (e.g., rational) number in
      $V_\epsilon(1)$. $\N$ is closed since it has no limit points.
    \item $\{x \in \R \st x \neq 0\}$ is open since it is the union of two open
      sets $(-\infty, 0)$ and $(0, \infty)$. $\{x \in \R \st x \neq 0\}$ is
      closed since it contains its limit points.
    \item $\{1 + 1/4 + 1/9 + \cdots + 1/n^2 \st n \in \N\}$ is not open. For
      example, no $V_\epsilon(1)$ is contained in the set since we can always
      find an irrational number in $V_\epsilon(1)$. $\{1 + 1/4 + 1/9 + \cdots +
      1/n^2 \st n \in \N\}$ is not closed since $\pi^2/6$ is a limit point that
      is not in the set.
    \item $\{1 + 1/2 + 1/3 + \cdots + 1/n \st n \in \N\}$ is not open. For
      example, no $V_\epsilon(1)$ is contained in the set since we can always
      find an irrational number in $V_\epsilon(1)$. $\{1 + 1/2 + 1/3 + \cdots +
      1/n \st n \in \N\}$ is closed since it does not contain any limit points.
  \end{enumerate}
\end{exercise}

\end{document}
