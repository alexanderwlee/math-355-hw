\documentclass{amsart}

\usepackage{enumitem}
\usepackage{mathtools}

\theoremstyle{definition}
\newtheorem{exercise}{Exercise}

\DeclarePairedDelimiter\abs{\lvert}{\rvert} % absolute value
% Swap the definition of \abs*, so that \abs
% resizes the size of the bars, and the starred version does not.
\makeatletter
\let\oldabs\abs%
\def\abs{\@ifstar{\oldabs}{\oldabs*}}
\makeatother

\newcommand{\N}{\mathbb{N}}
\newcommand{\Z}{\mathbb{Z}}
\newcommand{\Q}{\mathbb{Q}}
\newcommand{\I}{\mathbb{I}}
\newcommand{\R}{\mathbb{R}}
\newcommand{\card}[1]{\abs{#1}}
\newcommand{\st}{\mathrel{:}}

\title{MATH 355: Homework 5}
\author{Alexander Lee}

\begin{document}

\maketitle

\begin{exercise}[2.5.1]
  \begin{enumerate}[label={(\alph*)}]
    \item TODO % is the subsequence of sequence bounded?
    \item TODO
    \item TODO
    \item TODO
  \end{enumerate}
\end{exercise}

\begin{exercise}[2.5.2]
  \begin{enumerate}[label={(\alph*)}]
    \item TODO % what is a proper subsequence?
    \item True. Suppose towards a contradiction that $(x_n)$ converges. Thus, we
      know that subsequences of $(x_n)$ are convergent, which contradicts the
      assumption that $(x_n)$ contains a divergent subsequence. Therefore,
      $(x_n)$ diverges.
    \item TODO
    \item TODO
  \end{enumerate}
\end{exercise}

\begin{exercise}[2.6.2]
  \begin{enumerate}[label={(\alph*)}]
    \item Consider the sequence $a_n = {(-1)}^n \frac{1}{n}$. $(a_n)$ clearly
      converges to 0, so it is a Cauchy sequence by the Cauchy Criterion. Notice
      that $(a_n)$ is not monotone as well.
    \item Impossible. Suppose $(a_n)$ is a Cauchy sequence. Thus, $(a_n)$ is
      bounded by Lemma 2.6.3. As such, every subsequence of $(a_n)$ must be
      bounded as well.
    \item TODO
    \item Consider the sequence $(a_n) = (0, 1, 0, 2, 0, 3, \ldots)$. Clearly,
      $(a_n)$ is unbounded. However, $(0, 0, 0, \ldots)$ is a subsequence of
      $(a_n)$ that is Cauchy.
  \end{enumerate}
\end{exercise}

\begin{exercise}[2.6.4]
  \begin{enumerate}[label={(\alph*)}]
    \item Let $\epsilon > 0$ be arbitrary. Since $(a_n)$ is a Cauchy sequence,
      there exists an $N_1 \in \N$ such that whenever $m, n \ge N_1$, it follows
      that $\abs{a_n - a_m} < \epsilon / 2$. Similarly, since $(b_n)$ is a
      Cauchy sequence, there exists an $N_2 \in \N$ such that whenever $m, n \ge
      N_2$, it follows that $\abs{b_n - b_m} < \epsilon / 2$. Let $N =
      \max\{N_1, N_2\}$ and suppose $m, n \ge N$. Then,
      \begin{align*}
        \abs{c_n - c_m} &= \abs{\abs{a_n - b_n} - \abs{a_m - b_m}} \\
        &\le \abs{\abs{(a_n - b_n) - (a_m - b_m)}}\ \text{(by the Triangle
        Inequality)} \\
        &= \abs{(a_n - a_m) + (b_m - b_n)} \\
        &\le \abs{a_n - a_m} + \abs{b_m - b_n}\ \text{(by the Triangle
        Inequality)} \\
        &< \epsilon / 2 + \epsilon / 2 \\
        &= \epsilon.
      \end{align*}
      Therefore, $(c_n)$ is a Cauchy sequence.
    \item Let $a_n = 1$. $(a_n)$ is a Cauchy sequence, but $c_n = {(-1)}^n$ is
      divergent and thus not a Cauchy sequence.
    \item Let $a_n = {(-1)}^n \frac{1}{n}$. Then,
      \[
        c_n =
        \begin{cases}
          0 &\text{if $n$ is odd} \\
          -1 &\text{if $n$ is even}
        \end{cases},
      \]
      which is divergent and thus not a Cauchy sequence.
  \end{enumerate}
\end{exercise}

\begin{exercise}[2.7.1]
  \begin{enumerate}[label={(\alph*)}]
    \item Let $\epsilon > 0$ be arbitrary. Since $(a_n) \rightarrow 0$, there
      exists an $N \in \N$ such that whenever $m \ge N$, it follows that
      $\abs{a_m} < \epsilon$. Then,
      \begin{align*}
        \abs{s_n - s_m} &= \abs{a_{m+1} - a_{m+2} - \cdots \pm a_n} \\
        &\le \abs{a_{m+1}}\ \text{(since $(a_n)$ is decreasing)} \\
        &\le \abs{a_m} \\
        &< \epsilon
      \end{align*}
      whenever $n > m \ge N$.
  \end{enumerate}
\end{exercise}

\begin{exercise}[2.7.2]
\end{exercise}

\begin{exercise}[2.7.4]
\end{exercise}

\begin{exercise}[2.7.5]
\end{exercise}

\begin{exercise}[2.7.9]
\end{exercise}

\begin{exercise}[3.2.3]
\end{exercise}

\end{document}
