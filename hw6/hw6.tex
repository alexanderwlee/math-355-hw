\documentclass{amsart}

\usepackage{enumitem}
\usepackage{mathtools}

\theoremstyle{definition}
\newtheorem{exercise}{Exercise}

\DeclarePairedDelimiter\abs{\lvert}{\rvert} % absolute value
% Swap the definition of \abs*, so that \abs
% resizes the size of the bars, and the starred version does not.
\makeatletter
\let\oldabs\abs%
\def\abs{\@ifstar{\oldabs}{\oldabs*}}
\makeatother

\newcommand{\N}{\mathbb{N}}
\newcommand{\Z}{\mathbb{Z}}
\newcommand{\Q}{\mathbb{Q}}
\newcommand{\I}{\mathbb{I}}
\newcommand{\R}{\mathbb{R}}
\newcommand{\card}[1]{\abs{#1}}
\newcommand{\st}{\mathrel{:}}

\title{MATH 355: Homework 6}
\author{Alexander Lee}

\begin{document}

\maketitle

\begin{exercise}[3.2.2]
  \begin{enumerate}[label={(\alph*)}]
    \item Limit points of $A$: $\{-1, 1\}$. Limit points of $B$: $[0, 1]$.
    \item $A$ is neither open nor closed. $B$ is neither open nor closed.
    \item $A$ contains isolated points. $B$ does not contain isolated points.
    \item $\overline{A} = A \cup \{-1\}$. $\overline{B} = [0, 1]$.
  \end{enumerate}
\end{exercise}

\begin{exercise}[3.2.4]
  \begin{enumerate}[label={(\alph*)}]
    \item If $s \in A$, then $s \in \overline{A}$ and we are done. Now suppose
      $s \notin A$. By Lemma 1.3.8, for every $\epsilon > 0$, there exists an $a
      \in A$ ($a \neq s$) such that $s - \epsilon < a$. Since $s = \sup(A)$, we
      also know that $a < s$. Thus, every $\epsilon$-neighborhood
      $V_\epsilon(s)$ intersects $A$ at some point other than $s$. That is, $s$
      is a limit point of $A$, so $s \in \overline{A}$ in this case as well.
    \item An open set $O$ cannot contain its supremum $s = \sup(O)$ since every
      $\epsilon$-neighborhood $V_\epsilon(s)$ of $s$ is not be a subset of $O$.
      Specifically, this is because for any $\epsilon > 0$ and $a \in O$, we
      have that $a < s + \epsilon$ since $s = \sup(O)$.
  \end{enumerate}
\end{exercise}

\begin{exercise}[3.2.6]
  \begin{enumerate}[label={(\alph*)}]
    \item False. Consider the open set $\R \setminus \{\sqrt{2}\}$.
    \item False. Consider the closed sets of the form $C_n = \lbrack n, \infty
      \rparen$ for $n \in \N$. Observe that $C_n \subseteq C_{n+1}$ and
      $\cup_{n=1}^\infty C_n = \emptyset$.
    \item True. Given a nonempty open set $O$, we know that for $a \in O$, there
      exists an $\epsilon$-neighborhood $V_\epsilon(a) \subseteq O$. By the
      Density of $\Q$ in $\R$, there exists a rational number $r \in
      V_\epsilon(a)$. Thus, we have that $r \in O$.
    \item False. Consider the bounded infinite closed set $F = \{\sqrt{2} + 1/n
      \st n \in \N\} \cup \{\sqrt{2}\}$. Observe that $F$ does not contain any
      rational number.
    \item True. The Canter set is defined as $C = \cap_{n=0}^\infty C_n$. Since
      each $C_n$ is the union of a finite collection of closed sets, each $C_n$
      is closed. The intersection of an arbitrary collection of closed sets is
      closed, so the Cantor set $C$ is closed.
  \end{enumerate}
\end{exercise}

\begin{exercise}[3.2.8]
  \begin{enumerate}[label={(\alph*)}]
    \item Definitely closed.
    \item Definitely open. % $A \setminus B = A \cap B^c$
    \item Definitely open.
    \item Both. % $(A \cap B) \cup (A^c \cap B) = \R$
    \item Neither.
  \end{enumerate}
\end{exercise}

\begin{exercise}[3.2.9]
  \begin{enumerate}[label={(\alph*)}]
    \item We first show that ${(\cup_{\lambda \in \Lambda} E_\lambda)}^c =
      \cap_{\lambda \in \Lambda} E_\lambda^c$.
      \begin{align*}
        x \in {(\cup_{\lambda \in \Lambda} E_\lambda)}^c &\iff x \notin
        \cup_{\lambda \in \Lambda} E_\lambda \\
        &\iff \forall \lambda \in \Lambda,\ x \notin E_\lambda \\
        &\iff \forall \lambda \in \Lambda,\ x \in E_\lambda^c \\
        &\iff x \in \cap_{\lambda \in \Lambda} E_\lambda^c.
      \end{align*}
      Next, we show that ${(\cap_{\lambda \in \Lambda} E_\lambda)}^c =
      \cup_{\lambda \in \Lambda} E_\lambda^c$.
      \begin{align*}
        x \in {(\cap_{\lambda \in \Lambda} E_\lambda)}^c &\iff x \notin
        \cap_{\lambda \in \Lambda} E_\lambda \\
        &\iff \exists \lambda \in \Lambda\ \text{s.t.}\ x \notin E_\lambda \\
        &\iff \exists \lambda \in \Lambda\ \text{s.t.}\ x \in E_\lambda^c \\
        &\iff x \in \cup_{\lambda \in \Lambda} E_\lambda^c.
      \end{align*}
    \item Suppose $\{E_\lambda \st \lambda \in \Lambda\}$ is a finite collection
      of open sets. Each $E_\lambda^c$ is thus a closed set by Theorem 3.2.13.
      As such, $\cup_{\lambda \in \Lambda} E_\lambda^c$ is the union of a finite
      collection of closed sets. By Theorem 3.2.3, $\cap_{\lambda \in \Lambda}
      E_\lambda$ is open. It follows that ${(\cap_{\lambda \in \Lambda}
      E_\lambda)}^c$ is closed by Theorem 3.2.13. Since ${(\cap_{\lambda \in
      \Lambda} E_\lambda)}^c= \cup_{\lambda \in \Lambda} E_\lambda^c$, the union
      of a finite collection of closed sets is therefore closed.

      Suppose $\{E_\lambda \st \lambda \in \Lambda\}$ is an arbitrary
      collection of open sets. As such, $\cap_{\lambda \in \Lambda} E_\lambda^c$
      is the intersection of an arbitrary collection of closed sets. By Theorem
      3.2.3, $\cup_{\lambda \in \Lambda} E_\lambda$ is open. It follows that
      ${(\cup_{\lambda \in \Lambda} E_\lambda)}^c$ is closed by Theorem 3.2.13.
      Since ${(\cup_{\lambda \in \Lambda} E_\lambda)}^c = \cap_{\lambda \in
      \Lambda} E_\lambda^c$, the intersection of an arbitrary collection of
      closed sets is closed.
  \end{enumerate}
\end{exercise}

\begin{exercise}[3.2.10]
  \begin{enumerate}[label={(\roman*)}]
    \item Such a set cannot exist. Let $A \subseteq [0, 1]$ be a countable set.
      Since $A$ is countable, there exists a bijection $f : \N \to A$. We can
      use the function $f$ to define a sequence $(a_n)$ where $a_n = f(n)$ for
      all $n \in \N$. Because $(a_n) \subseteq A \subseteq [0, 1]$, $(a_n)$ is
      bounded. By the Bolzano-Weierstrass Theorem, $(a_n)$ has a convergent
      subsequence $(a_{n_k}) \to a$. Since $f$ is a bijection, all the terms of
      $(a_n)$ are distinct, so at most one term in $(a_{n_k})$ can be equal to
      $a$. Let $(b_n)$ be a subsequence of $(a_{n_k})$ without the term $a$ if
      it exists. It follows that $(b_n) \to a$, so $a$ is a limit point.
    \item Consider the set $\Q \cap [0, 1]$.
    \item Such a set cannot exist. Suppose $A$ has an uncountable number of
      isolated points. For every isolated point $x \in A$, there exists
      an $\epsilon_x > 0$ such that $V_{\epsilon_x}(x) \cap A = \{x\}$. Each
      neighborhood $V_{\epsilon_x}(x)$ can intersect with at most 2 other
      neighborhoods, since if a neighborhood intersects with more than 2
      neighborhoods, then one of the three neighborhoods in question  would
      intersect another isolated point in $A$, which is not possible. By the
      Density of $\Q$ in $\R$, we can choose a rational number $r$ in each
      neighborhood. Since each neighborhood intersects with at most 2 other
      neighborhoods, the same $r$ can be chosen for at most 3 neighborhoods.
      Regardless, we are still have uncountably many rational numbers since each
      rational number $r$ corresponds to at most 3 neighborhoods and we have
      uncountably many neighborhoods. This is a contradiction since $\Q$ is
      countable.
  \end{enumerate}
\end{exercise}

\begin{exercise}[3.2.13]
  TODO
\end{exercise}

\begin{exercise}[3.2.14]
  \begin{enumerate}[label={(\alph*)}]
    \item We first show that $E$ is closed if and only if $\overline{E} = E$.
      Let $L$ be the set of all limit points of $E$. Then,
      \begin{align*}
        E\ \text{is closed} &\iff L \subseteq E \\
        &\iff E \cup L = E \\
        &\iff \overline{E} = E.
      \end{align*}
      Next, we show that $E$ is open if and only if $E^\circ = E$.
      \begin{align*}
        E\ \text{is open} &\iff \forall x \in E,\ \exists V_\epsilon(x)
        \subseteq E \\
        &\iff E^\circ = E.
      \end{align*}
    \item We begin with showing that $\overline{E}^c = {(E^c)}^\circ$. Let $L$
      be the set of all limit points of $E$. Then,
      \begin{align*}
        x \in \overline{E}^c &\iff x \in {(E \cup L)}^c \\
        &\iff x \in E^c \cap L^c\ \text{(by DeMorgan's Laws)} \\
        &\iff x \in E^c \wedge x\ \text{is not a limit point of E} \\
        &\iff x \in E^c\ \text{s.t.}\ \exists V_\epsilon(x) \subseteq E^c \\
        &\iff x \in {(E^c)}^\circ.
      \end{align*}
      Next, we show that ${(E^\circ)}^c = \overline{E^c}$.
      \begin{align*}
        {(E^\circ)}^c &= {({({(E^c)}^c)}^\circ)}^c \\
        &= {(\overline{E^c}^c)}^c\ \text{(since $\overline{E}^c =
        {(E^c)}^\circ$)} \\
        &= \overline{E^c}.
      \end{align*}
  \end{enumerate}
\end{exercise}

\end{document}
