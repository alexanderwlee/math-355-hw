\documentclass{amsart}

\usepackage{enumitem}
\usepackage{mathtools}

\theoremstyle{definition}
\newtheorem{exercise}{Exercise}

\DeclarePairedDelimiter\abs{\lvert}{\rvert} % absolute value
% Swap the definition of \abs*, so that \abs
% resizes the size of the bars, and the starred version does not.
\makeatletter
\let\oldabs\abs%
\def\abs{\@ifstar{\oldabs}{\oldabs*}}
\makeatother

\newcommand{\N}{\mathbb{N}}
\newcommand{\Z}{\mathbb{Z}}
\newcommand{\Q}{\mathbb{Q}}
\newcommand{\I}{\mathbb{I}}
\newcommand{\R}{\mathbb{R}}
\newcommand{\card}[1]{\abs{#1}}
\newcommand{\st}{\mathrel{:}}

\title{MATH 355: Homework 6}
\author{Alexander Lee}

\begin{document}

\maketitle

\begin{exercise}[3.2.2]
  \begin{enumerate}[label={(\alph*)}]
    \item Limit points of $A$: $\{-1, 1\}$. Limit points of $B$: $[0, 1]$.
    \item $A$ is neither open nor closed. $B$ is neither open nor closed.
    \item $A$ contains isolated points. $B$ does not contain isolated points.
    \item $\overline{A} = A \cup \{-1\}$. $\overline{B} = [0, 1]$.
  \end{enumerate}
\end{exercise}

\begin{exercise}[3.2.4]
  \begin{enumerate}[label={(\alph*)}]
    \item If $s \in A$, then $s \in \overline{A}$ and we are done. Now suppose
      $s \notin A$. By Lemma 1.3.8, for every $\epsilon > 0$, there exists an $a
      \in A$ ($a \neq s$) such that $s - \epsilon < a$. Since $s = \sup(A)$, we
      also know that $a < s$. Thus, every $\epsilon$-neighborhood
      $V_\epsilon(s)$ intersects $A$ at some point other than $s$. That is, $s$
      is a limit point of $A$, so $s \in \overline{A}$ in this case as well.
    \item An open set $O$ cannot contain its supremum $s = \sup(O)$ since every
      $\epsilon$-neighborhood $V_\epsilon(s)$ of $s$ is not be a subset of $O$.
      Specifically, this is because for any $\epsilon > 0$ and $a \in O$, we
      have that $a < s + \epsilon$ since $s = \sup(O)$.
  \end{enumerate}
\end{exercise}

\begin{exercise}[3.2.6]
\end{exercise}

\begin{exercise}[3.2.8]
\end{exercise}

\begin{exercise}[3.2.9]
\end{exercise}

\begin{exercise}[3.2.10]
\end{exercise}

\begin{exercise}[3.2.13]
\end{exercise}

\begin{exercise}[3.2.14]
\end{exercise}

\end{document}
