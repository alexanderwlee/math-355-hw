\documentclass{amsart}

\usepackage{enumitem}
\usepackage{mathtools}

\theoremstyle{definition}
\newtheorem{exercise}{Exercise}

\DeclarePairedDelimiter\abs{\lvert}{\rvert} % absolute value
% Swap the definition of \abs*, so that \abs
% resizes the size of the bars, and the starred version does not.
\makeatletter
\let\oldabs\abs%
\def\abs{\@ifstar{\oldabs}{\oldabs*}}
\makeatother

\newcommand{\N}{\mathbb{N}}
\newcommand{\Z}{\mathbb{Z}}
\newcommand{\Q}{\mathbb{Q}}
\newcommand{\I}{\mathbb{I}}
\newcommand{\R}{\mathbb{R}}
\newcommand{\card}[1]{\abs{#1}}
\newcommand{\st}{\mathrel{:}}

\title{MATH 355: Homework 7}
\author{Alexander Lee}

\begin{document}

\maketitle

\begin{exercise}[3.3.1]
  Since $K$ is compact, it is closed and bounded. It follows from the Axiom of
  Completeness that because $K$ is nonempty and bounded (above and below),
  $\sup(K)$ and $\inf(K)$ both exist. Since $K$ is closed, $K = \overline{K}$,
  so $\sup(K) \in \overline{K} = K$ by Exercise 3.2.4 (a). By similar reasoning,
  it follows that $\inf(K) \in K$.
\end{exercise}

\begin{exercise}[3.3.2, except (c)]
  \begin{enumerate}[label={(\alph*)}]
    \item Not compact. Consider the sequence $(a_n)$ where $a_n = n$.
    \item Not compact. We can construct a sequence $(a_n)$ that converges to $1
      / \sqrt{2} \notin \Q \cap [0, 1]$. All subsequences of $(a_n)$ must also
      converge to $1 / \sqrt{2}$.
    \addtocounter{enumi}{1}
    \item Not compact. Consider the sequence of partial sums $(s_m)$ where $s_m
      = \sum_{n=1}^m 1 / n^2$.
    \item Compact. The set is closed since 1 is its only limit point, which is
      contained in the set, and the set is also bounded.
  \end{enumerate}
\end{exercise}

\begin{exercise}[3.3.4]
  \begin{enumerate}[label={(\alph*)}]
    \item The set is definitely closed since $K$ is compact and thus closed,
      and the intersection of an arbitrary collection of closed sets is closed.
      The set is definitely compact since $K \cap F$ is closed as explained
      previously and is bounded because $K$ is bounded.
    \item The set is definitely closed since the closure of any set is closed.
      The set is not definitely compact.
    \item Neither.
    \item The set is definitely closed since the closure of any set is closed.
      The set is definitely compact since it is closed as explained previously
      and is bounded since $K \cap F^c$ is bounded and the closure of a bounded
      set is also bounded.
  \end{enumerate}
\end{exercise}

\begin{exercise}[3.3.5]
  \begin{enumerate}[label={(\alph*)}]
    \item True. The arbitrary intersection of compact sets is closed since each
      compact set is closed and the arbitrary intersection of closed sets is
      closed. The arbitrary intersection of compact sets is bounded since each
      compact set is bounded and the intersection of bounded sets must be
      bounded.
    \item False. Consider $\cup_{n=1}^\infty [n, n+1]$. This union of compact
      sets is unbounded, and thus not compact.
    \item False. Consider $A = (0, 1)$ and $K = [0, 1]$. $A \cap K = (0, 1)$ is
      open, and thus not compact.
    \item False. Consider $\cap_{n=1}^\infty \lbrack n, \infty \rparen =
      \emptyset$.
  \end{enumerate}
\end{exercise}

\begin{exercise}[3.3.11]
  $\N$: consider the open cover $\{V_1(n) \st n \in \N\}$.

  $\Q \cap [0, 1]$: consider the open cover TODO

  $\{1 + 1/2^2 + 1/3^2 + \cdots + 1/n^2 \st n \in \N\}$: consider the open cover
  TODO
\end{exercise}

\begin{exercise}[4.2.2]
  \begin{enumerate}[label={(\alph*)}]
    \item TODO
    \item TODO
    \item TODO
    \item TODO
  \end{enumerate}
\end{exercise}

\begin{exercise}[4.2.5]
  \begin{enumerate}[label={(\alph*)}]
    \item Let $\epsilon > 0$. Notice that
      \[
        \abs{3x + 4 - 10} = \abs{3x + 6} = 3 \abs{x - 2}.
      \]
      Thus, if we choose $\delta = \epsilon / 3$, then $0 < \abs{x - 2} <
      \delta$ implies $\abs{3x + 4 - 10} < 3 (\epsilon / 3) = \epsilon$.
    \item Let $\epsilon > 0$. Notice that
      \[
        \abs{x^3 - 0} = \abs{x^3} = x^2 \abs{x}.
      \]
      Choose a $\delta$-neighborhood around $c = 0$ to with radius no bigger
      than $\delta = 1$. Then, we get the upper bound $x^2 \le 1$ for all $x \in
      V_\delta(c)$. Now, choose $\delta = \min\{1, \epsilon\}$. If $0 < \abs{x -
      0} < \delta$, then it follows that
      \[
        \abs{x^3 - 0} = x^2 \abs{x} < 1 (\epsilon) = \epsilon.
      \]
    \item Let $\epsilon > 0$. Notice that
      \[
        \abs{x^2 + x - 1 - 5} = \abs{x^2 + x - 6} = \abs{x + 3} \abs{x - 2}.
      \]
      Choose a $\delta$-neighborhood around $c = 2$ with radius no bigger than
      $\delta = 1$. Then, we get the upper bound $\abs{x + 3} \ge \abs{3 + 3} =
      6$ for all $x \in V_\delta(c)$. Now, choose $\delta = \min\{1, \epsilon /
      6\}$. If $0 < \abs{x - 2} < \delta$, then it follows that
      \[
        \abs{x^3 + x - 1 - 5} = \abs{x + 3} \abs{x - 2} < 6 (\epsilon / 6) =
        \epsilon.
      \]
    \item Let $\epsilon > 0$. Notice that
      \[
        \abs{1/x - 1/3} = \abs{\frac{3 - x}{3x}} = \frac{\abs{3 - x}}{\abs{3x}}.
      \]
      Choose a $\delta$-neighborhood around $c = 3$ with radius no bigger than
      $\delta = 1$. Then, we get the lower bound $\abs{3x} \ge \abs{3 \cdot 2} =
      6$. Now, choose $\delta = \min\{1, 6 / \epsilon\}$. If $0 < \abs{x - 3} <
      \delta$, then it follows that
      \[
        \abs{1/x - 1/3} = \frac{\abs{3 - x}}{\abs{3x}} < \frac{6}{\epsilon}
        \cdot \frac{1}{6} = \epsilon.
      \]
  \end{enumerate}
\end{exercise}

\begin{exercise}[4.2.6]
  \begin{enumerate}[label={(\alph*)}]
    \item True. If a particular $\delta$ has been constructed as a suitable
      response to a particular $\epsilon$ challenge, then we know that for all
      $x \in V_\delta(c)$ different from $c$ (with $x \in A$) it follows that
      $f(x) \in V_\epsilon(L)$. Therefore, any smaller positive $\delta$ will
      also suffice since all $x$'s will still be in the $\delta$-neighborhood.
    \item True. TODO
    \item True. Given $\lim_{x \to a} f(x) = L$, we have that
      \[
        \lim_{x \to a} 3{[f(x) - 2]}^2 = 3{[\lim_{x \to a} f(x) - 2]}^2 = 3{(L -
        2)}^2
      \]
      by the Algebraic Limit Theorem for Functional Limits.
    \item False. $\lim_{x \to a} g(x)$ may not exist.
  \end{enumerate}
\end{exercise}

\end{document}
