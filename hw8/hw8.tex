\documentclass{amsart}

\usepackage{enumitem}
\usepackage{mathtools}

\theoremstyle{definition}
\newtheorem{exercise}{Exercise}

\DeclarePairedDelimiter\abs{\lvert}{\rvert} % absolute value
% Swap the definition of \abs*, so that \abs
% resizes the size of the bars, and the starred version does not.
\makeatletter
\let\oldabs\abs%
\def\abs{\@ifstar{\oldabs}{\oldabs*}}
\makeatother

\newcommand{\N}{\mathbb{N}}
\newcommand{\Z}{\mathbb{Z}}
\newcommand{\Q}{\mathbb{Q}}
\newcommand{\I}{\mathbb{I}}
\newcommand{\R}{\mathbb{R}}
\newcommand{\card}[1]{\abs{#1}}
\newcommand{\st}{\mathrel{:}}

\title{MATH 355: Homework 8}
\author{Alexander Lee}

\begin{document}

\maketitle

\begin{exercise}[4.2.10]
  \begin{enumerate}[label={(\alph*)}]
    \item (Right-hand limit) Let $f : A \to \R$, and let $A$ be a limit point of
      the domain $A$. We say that $\lim_{x \to a^+} f(x) = L$ provided that, for
      all $\epsilon > 0$, there exists a $\delta > 0$ such that whenever $0 < a
      - x < \delta$ (and $x \in A$) it follows that $\abs{f(x) - L} < \epsilon$.

      (Left-hand limit) Let $f : A \to \R$, and let $A$ be a limit point of the
      domain $A$. We say that $\lim_{x \to a^-} f(x) = M$ provided that, for all
      $\epsilon > 0$, there exists a $\delta > 0$ such that whenever $0 < x - a
      < \delta$ (and $x \in A$) it follows that $\abs{f(x) - M} < \epsilon$.
    \item ($\implies$) Suppose that $\lim_{x \to a} f(x) = L$. By the definition
      of a functional limit, for all $\epsilon > 0$, there exists a $\delta > 0$
      such that whenever $0 < \abs{x - c} < \delta$ (and $x \in A$) it follows
      that $\abs{f(x) - L} < \epsilon$. Thus, for this chosen $\delta$, we have
      that $0 < c - x < \delta$ (and $x \in A$) implies $\abs{f(x) - L} <
      \epsilon$, and $0 < x - c < \delta$ (and $x \in A$) implies $\abs{f(x) -
      L} < \epsilon$. Therefore, $\lim_{x \to c^+} f(x) = \lim_{x \to c^-} f(x)
      = L$ (i.e., both the right and left-hand limits equal $L$).

      ($\impliedby$) Suppose $\lim_{x \to c^+} f(x) = \lim_{x \to c^-} f(x) =
      L$. Since we have that $\lim_{x \to c^+} f(x) = L$, for all $\epsilon >
      0$, there exists a $\delta_1 > 0$ such that $0 < c - x < \delta_1$ (and $x
      \in A$) implies $\abs{f(x) - L} < \epsilon$. Similarly, since we have that
      $\lim_{x \to c^-} f(x) = L$, for all $\epsilon > 0$, there exists a
      $\delta_2 > 0$ such that $0 < x - c < \delta_2$ (and $x \in A$) implies
      $\abs{f(x) - L} < \epsilon$. Let $\delta = \min\{\delta_1, \delta_2\}$.
      Thus, for all $\epsilon > 0$, we have that $0 < c - x < \delta \le
      \delta_1$ (and $x \in A$) implies $\abs{f(x) - L} < \epsilon$ and $0 < x -
      c < \delta \le \delta_2$ (and $x \in A$) implies $\abs{f(x) - L} <
      \epsilon$. It follows immediately that $0 < \abs{x - c} < \delta$ (and $x
      \in A$) implies $\abs{f(x) - L} < \epsilon$. Therefore, $\lim_{x \to a}
      f(x) = L$.
  \end{enumerate}
\end{exercise}

\begin{exercise}[4.2.11]
  Since $\lim_{x \to c} f(x) = L$ and $\lim_{x \to c} h(x) = L$, by the
  Sequential Criterion for Functional Limits, we know that for all sequences
  $(x_n) \subseteq A$ satisfying $x_n \neq c$ and $\lim x_n = c$, it follows
  that $\lim f(x_n) = L$ and $\lim h(x_n) = L$. By assumption, we have $f(x_n)
  \le g(x_n) \le h(x_n)$. Applying the Squeeze Theorem for sequences, it follows
  that $\lim g(x_n) = L$, implying that $\lim_{x \to c} g(x) = L$ as well.
\end{exercise}

\begin{exercise}[4.3.2]
  \begin{enumerate}[label={(\alph*)}]
    \item Consider the function $f(x) = k$, for some $k \in \R$.
    \item Consider the function $f(x) = x$.
    \item Consider the function $f(x) = 2x$.
    \item Every lesstinuous function is continuous, since choosing $0 < \delta <
      \epsilon$ implies choosing a $\delta > 0$. Every continuous function is
      also lesstinuous, since if we choose a $\delta$, where $0 < \epsilon \le
      \delta$, which satisfies the definition for continuity, we can choose a
      $\delta'$ such that $0 < \delta' < \epsilon \le \delta$.
  \end{enumerate}
\end{exercise}

\begin{exercise}[4.3.5]
  Suppose $c$ is an isolated point of $A \subseteq \R$. Let $\epsilon > 0$ be
  arbitrary. Since $c$ is an isolated point, there exists a
  $\delta$-neighborhood $V_\delta(c)$ of $c$ that only intersects $A$ at $c$.
  That is, there exists a $\delta > 0$ such that the only $x \in A$ where
  $\abs{x - c} < \delta$ is $x = c$. Thus, with the chosen $\delta$, whenever
  $\abs{x - c} < \delta$, it follows that $\abs{f(x) - f(c)} = \abs{f(c) - f(c)}
  = 0 < \epsilon$.
\end{exercise}

\begin{exercise}[4.3.6]
  \begin{enumerate}[label={(\alph*)}]
    \item Consider functions
      \[
        f(x) =
        \begin{cases}
          x + 1 & \text{if}\ x \neq 0 \\
          0 & \text{if}\ x = 0
        \end{cases}
        \quad \text{and} \quad
        g(x) =
        \begin{cases}
          0 & \text{if}\ x \neq 0 \\
          1 & \text{if}\ x = 0
        \end{cases},
      \]
      neither of which is continuous at 0.
      We have that $f(x) g(x) = 0$ and $f(x) + g(x) = x + 1$, both of which are
      continuous at 0.
    \item Impossible. Given that $f(x)$ and $f(x) + g(x)$ are continuous at
      0, $g(x) = [f(x) + g(x)] - f(x)$ must also be continuous at 0 by the
      Algebraic Continuity Theorem.
    \item Consider functions
      \[
        f(x) = x
        \quad \text{and} \quad
        g(x) =
        \begin{cases}
          0 & \text{if}\ x \neq 0 \\
          1 & \text{if}\ x = 0
        \end{cases}.
      \]
      Observe that $f(x)$ is continuous at 0 and $g(x)$ is not continuous at 0.
      However, we have that $f(x) g(x) = 0$, which is continuous at 0.
    \item Consider the function
      \[
        f(x) =
        \begin{cases}
          2 + \sqrt{3} & \text{if}\ x \neq 0 \\
          2 - \sqrt{3} & \text{if}\ x = 0
        \end{cases},
      \]
      which is not continuous at 0. When $x \neq 0$, it holds that
      \begin{align*}
        f(x) + \frac{1}{f(x)} &= 2 + \sqrt{3} + \frac{1}{2 + \sqrt{3}} \\
        &= 2 + \sqrt{3} + \frac{1}{2 + \sqrt{3}} \cdot \frac{2 - \sqrt{3}}{2 -
        \sqrt{3}} \\
        &= 2 + \sqrt{3} + 2 - \sqrt{3} \\
        &= 4.
      \end{align*}
      Similarly, when $x = 0$, it holds that
      \begin{align*}
        f(x) + \frac{1}{f(x)} &= 2 - \sqrt{3} + \frac{1}{2 - \sqrt{3}} \\
        &= 2 - \sqrt{3} + \frac{1}{2 - \sqrt{3}} \cdot \frac{2 + \sqrt{3}}{2 +
        \sqrt{3}} \\
        &= 2 - \sqrt{3} + 2 + \sqrt{3} \\
        &= 4.
      \end{align*}
      Therefore, we have that $f(x) + \frac{1}{f(x)} = 4$, which is continuous
      at 0.
    \item TODO
  \end{enumerate}
\end{exercise}

\begin{exercise}[4.3.8]
\end{exercise}

\begin{exercise}[4.4.11]
\end{exercise}

\begin{exercise}[4.4.12]
\end{exercise}

\end{document}
