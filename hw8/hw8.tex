\documentclass{amsart}

\usepackage{enumitem}
\usepackage{mathtools}

\theoremstyle{definition}
\newtheorem{exercise}{Exercise}

\DeclarePairedDelimiter\abs{\lvert}{\rvert} % absolute value
% Swap the definition of \abs*, so that \abs
% resizes the size of the bars, and the starred version does not.
\makeatletter
\let\oldabs\abs%
\def\abs{\@ifstar{\oldabs}{\oldabs*}}
\makeatother

\newcommand{\N}{\mathbb{N}}
\newcommand{\Z}{\mathbb{Z}}
\newcommand{\Q}{\mathbb{Q}}
\newcommand{\I}{\mathbb{I}}
\newcommand{\R}{\mathbb{R}}
\newcommand{\card}[1]{\abs{#1}}
\newcommand{\st}{\mathrel{:}}

\title{MATH 355: Homework 8}
\author{Alexander Lee}

\begin{document}

\maketitle

\begin{exercise}[4.2.10]
  \begin{enumerate}[label={(\alph*)}]
    \item (Right-hand limit) Let $f : A \to \R$, and let $A$ be a limit point of
      the domain $A$. We say that $\lim_{x \to a^+} f(x) = L$ provided that, for
      all $\epsilon > 0$, there exists a $\delta > 0$ such that whenever $0 < a
      - x < \delta$ (and $x \in A$) it follows that $\abs{f(x) - L} < \epsilon$.

      (Left-hand limit) Let $f : A \to \R$, and let $A$ be a limit point of the
      domain $A$. We say that $\lim_{x \to a^-} f(x) = M$ provided that, for all
      $\epsilon > 0$, there exists a $\delta > 0$ such that whenever $0 < x - a
      < \delta$ (and $x \in A$) it follows that $\abs{f(x) - M} < \epsilon$.
    \item ($\implies$) Suppose that $\lim_{x \to a} f(x) = L$. By the definition
      of a functional limit, for all $\epsilon > 0$, there exists a $\delta > 0$
      such that whenever $0 < \abs{x - c} < \delta$ (and $x \in A$) it follows
      that $\abs{f(x) - L} < \epsilon$. Thus, for this chosen $\delta$, we have
      that $0 < c - x < \delta$ (and $x \in A$) implies $\abs{f(x) - L} <
      \epsilon$, and $0 < x - c < \delta$ (and $x \in A$) implies $\abs{f(x) -
      L} < \epsilon$. Therefore, $\lim_{x \to c^+} f(x) = \lim_{x \to c^-} f(x)
      = L$ (i.e., both the right and left-hand limits equal $L$).

      ($\impliedby$) Suppose $\lim_{x \to c^+} f(x) = \lim_{x \to c^-} f(x) =
      L$. Since we have that $\lim_{x \to c^+} f(x) = L$, for all $\epsilon >
      0$, there exists a $\delta_1 > 0$ such that $0 < c - x < \delta_1$ (and $x
      \in A$) implies $\abs{f(x) - L} < \epsilon$. Similarly, since we have that
      $\lim_{x \to c^-} f(x) = L$, for all $\epsilon > 0$, there exists a
      $\delta_2 > 0$ such that $0 < x - c < \delta_2$ (and $x \in A$) implies
      $\abs{f(x) - L} < \epsilon$. Let $\delta = \min\{\delta_1, \delta_2\}$.
      Thus, for all $\epsilon > 0$, we have that $0 < c - x < \delta \le
      \delta_1$ (and $x \in A$) implies $\abs{f(x) - L} < \epsilon$ and $0 < x -
      c < \delta \le \delta_2$ (and $x \in A$) implies $\abs{f(x) - L} <
      \epsilon$. It follows immediately that $0 < \abs{x - c} < \delta$ (and $x
      \in A$) implies $\abs{f(x) - L} < \epsilon$. Therefore, $\lim_{x \to a}
      f(x) = L$.
  \end{enumerate}
\end{exercise}

\begin{exercise}[4.2.11]
  Since $\lim_{x \to c} f(x) = L$ and $\lim_{x \to c} h(x) = L$, by the
  Sequential Criterion for Functional Limits, we know that for all sequences
  $(x_n) \subseteq A$ satisfying $x_n \neq c$ and $\lim x_n = c$, it follows
  that $\lim f(x_n) = L$ and $\lim h(x_n) = L$. By assumption, we have $f(x_n)
  \le g(x_n) \le h(x_n)$. Applying the Squeeze Theorem for sequences, it follows
  that $\lim g(x_n) = L$, implying that $\lim_{x \to c} g(x) = L$ as well.
\end{exercise}

\begin{exercise}[4.3.2]
\end{exercise}

\begin{exercise}[4.3.5]
\end{exercise}

\begin{exercise}[4.3.6]
\end{exercise}

\begin{exercise}[4.3.8]
\end{exercise}

\begin{exercise}[4.4.11]
\end{exercise}

\begin{exercise}[4.4.12]
\end{exercise}

\end{document}
