\documentclass{amsart}

\usepackage{enumitem}
\usepackage{mathtools}

\theoremstyle{definition}
\newtheorem{exercise}{Exercise}

\DeclarePairedDelimiter\abs{\lvert}{\rvert} % absolute value
% Swap the definition of \abs*, so that \abs
% resizes the size of the bars, and the starred version does not.
\makeatletter
\let\oldabs\abs%
\def\abs{\@ifstar{\oldabs}{\oldabs*}}
\makeatother

\newcommand{\N}{\mathbb{N}}
\newcommand{\Z}{\mathbb{Z}}
\newcommand{\Q}{\mathbb{Q}}
\newcommand{\I}{\mathbb{I}}
\newcommand{\R}{\mathbb{R}}
\newcommand{\card}[1]{\abs{#1}}
\newcommand{\st}{\mathrel{:}}

\title{MATH 355: Homework 9}
\author{Alexander Lee}

\begin{document}

\maketitle

\begin{exercise}[4.4.1]
  \begin{enumerate}[label={(\alph*)}]
    \item Given $c \in \R$, we have
      \[
        \abs{f(x) - f(c)} = \abs{x^3 - c^3} = \abs{x - c} \abs{x^2 + xc + c^2}.
      \]
      Choosing $\delta \le 1$, we thus have $x \in (c - 1, c + 1)$. Hence,
      \[
        \abs{x^2 + xc + c^2} < {(c + 1)}^2 + {(c + 1)}^2 + c^2 < 3{(c + 1)}^2.
      \]
      Now, let $\delta = \min\{1, \epsilon / {(3(c+1))}^2\}$. Then, $\abs{x - c}
      < \delta$ implies
      \[
        \abs{f(x) - f(c)} < \left(\frac{\epsilon}{3 {(c + 1)}^2}\right) 3{(c +
        1)}^2 = \epsilon.
      \]
    \item Choose $x_n = n$ and $y_n = n + 1/n$. Observe that $\abs{x_n - y_n} =
      1/n \to 0$ and
      \[
        \abs{f(x_n) - f(y_n)} = \abs{n^3 - {\left(n + \frac{1}{n}\right)}^3} =
        3n + \frac{3}{n} + \frac{1}{n^3} \ge 3.
      \]
    \item Suppose $A$ is bounded by $M$. Given $x, c \in A$, we have that
      $\abs{x^2 + xc + c^2} \le 3M^2$. For any $\epsilon > 0$, we can choose
      $\delta = \epsilon / {(3M)}^2$. If $\abs{x - c} < \delta$, it follows that
      \[
        \abs{f(x) - f(c)} < \left(\frac{\epsilon}{3M^2}\right) 3M^2 = \epsilon.
      \]
  \end{enumerate}
\end{exercise}

\begin{exercise}[4.4.3]
  Observe that
  \[
    \abs{f(x) - f(y)} = \abs{\frac{1}{x^2} - \frac{1}{y^2}} = \abs{\frac{y^2 -
    x^2}{x^2 y^2}} = \abs{y - x} \left(\frac{y + x}{x^2 y^2}\right).
  \]

  If $x, y \in \lbrack 1, \infty \rparen$, then we have
  \[
    \frac{y + x}{x^2 y^2} = \frac{1}{x^2 y} + \frac{1}{x y^2} \le 1 + 1 = 2.
  \]
  Given $\epsilon > 0$, let $\delta = \epsilon / 2$ and it follows that
  $\abs{f(x) - f(y)} < (\epsilon / 2) 2 = \epsilon$ whenever $\abs{x - y} <
  \delta$. Therefore, $f$ is uniformly continuous on $\lbrack 1, \infty
  \rparen$.

  If $x, y \in \lparen 0, 1 \rbrack$, then set $x_n = 1 / \sqrt{n}$ and $y_n =
  1 / \sqrt{n + 1}$. Then, $\abs{x_n - y_n} \to 0$ and
  \[
    \abs{f(x_n) - f(y_n)} = \abs{n - (n + 1)} = 1.
  \]
  By the Sequential Criterion for Absence of Uniform Continuity, $f$ is not
  continuous on $\lparen 0, 1 \rbrack$.
\end{exercise}

\begin{exercise}[4.4.7]
  We first show that $f(x) = \sqrt{x}$ is uniformly continuous on $\lbrack 1,
  \infty \rparen$. Let $x, y \in \lbrack 1, \infty \rparen$. It follows that
  \[
    \abs{f(x) - f(y)} = \abs{\sqrt{x} - \sqrt{y}} = \abs{\frac{x - y}{\sqrt{x}
    + \sqrt{y}}} \le \abs{x - y} \frac{1}{2}.
  \]
  Given $\epsilon > 0$, let $\delta = 2 \epsilon$. It follows that $\abs{f(x) -
  f(y)} < (2 \epsilon) \frac{1}{2} = \epsilon$ whenever $\abs{x - y} < \delta$.
  Thus, $f(x) = \sqrt{x}$ is uniformly continuous on $\lbrack 1, \infty
  \rparen$.

  We also know that $f(x) = \sqrt{x}$ is continuous on $[0, 1]$ and $[0, 1]$ is
  a compact set, so $f$ is also uniformly continuous on $[0, 1]$. By Exercise
  4.4.5, we thus conclude that $f$ is uniformly continuous on $\lbrack 0, \infty
  \rparen$.
\end{exercise}

\begin{exercise}[4.5.2]
  \begin{enumerate}[label={(\alph*)}]
    \item Consider
      \[
        f(x) =
        \begin{cases}
          -1 & x \in (-2, -1) \\
          x & x \in [-1, 1] \\
          1 & x \in (1, 2)
        \end{cases}.
      \]
      Observe that $f$ is continuous on the open interval $(-2, 2)$ and has
      range equal to the closed interval $[-1, 1]$.
    \item Impossible. If a continuous function is defined a closed interval,
      then the continuous function is defined on a compact set. By the
      Preservation of Compact Sets, the range must also be compact and thus
      cannot be an open interval.
    \item Consider
      \[
        f(x) =
        \begin{cases}
          1/x & x \in (0, 1) \\
          1 & x \in \lbrack 1, 2 \rparen
        \end{cases}.
      \]
      Observe that $f$ is continuous on the open interval $(0, 2)$ with range
      equal to the unbounded closed set $\lbrack 1, \infty \rparen \neq \R$.
    \item Impossible. Suppose towards a contradiction that $f$ is a continuous
      function defined on all of $\R$ with range equal to $\Q$. Then, there
      exists $a, b \in \R$ such that $f(a), f(b) \in \Q$. Without loss of
      generality, suppose $a < b$ and  $f(a) < f(b)$. By the Density of
      Irrationals in $\R$, there exists an irrational number $L$ such that $f(a)
      < L < f(b)$. By the Intermediate Value Theorem, there exists a point $c
      \in (a, b)$ where $f(c) = L$, which contradicts our assumption that $f$
      has range equal to $\Q$.
  \end{enumerate}
\end{exercise}

\begin{exercise}[4.5.4]
  If $g$ is one-to-one, there are no points where $g$ fails to be one-to-one.
  Thus, $F$ is empty in this case.

  If $g$ is not one-to-one, then there exist points $x, y \in A$, $x \neq y$,
  such that $f(x) = f(y)$. Without loss of generality, suppose $x < y$. If $f$
  is the constant function, we are done. Otherwise, choose $z \in (x, y)$ such
  that $f(z) \neq f(x)$. Without loss of generality, suppose $f(x) < f(z)$. By
  the Intermediate Value Theorem, for all $L \in (f(x), f(z))$, there exists a
  point $c_1 \in (x, z)$ where $f(c_1) = L$. Similarly, for all $L \in (f(x),
  f(z))$, there exists a point $c_2 \in (z, y)$ where $f(c_2) = L$. Therefore,
  for all $c_1 \in (x, z)$, there exists $c_2 \in (z, y)$ such that $f(c_1) =
  f(c_2)$. Since $(z, y)$ is countable, $F$ is uncountable as well.
\end{exercise}

\begin{exercise}[4.5.7]
  TODO
\end{exercise}

\begin{exercise}[5.2.2]
  \begin{enumerate}[label={(\alph*)}]
    \item Consider $f(x) = g(x) = \abs{x}$. Clearly, $f$ and $g$ are not
      differentiable at zero. However, $(fg)(x) = \abs{x} \cdot \abs{x} = x^2$,
      which is differentiable at zero.
    \item TODO
    \item Impossible. Given that $g$ and $f + g$ are differentiable at zero, $f
      = (f + g) - g$ is also differentiable at zero by the Algebraic
      Differentiability Theorem.
    \item TODO
  \end{enumerate}
\end{exercise}

\begin{exercise}[5.2.6]
  \begin{enumerate}[label={(\alph*)}]
    \item The new definition replaces $x$ from Definition 5.2.1 with $c + h$.
    \item
      \begin{align*}
        \lim_{h \to 0} \frac{g(c + h) - g(c - h)}{2h} &= \frac{1}{2} \left(
        \lim_{h \to 0} \frac{g(c + h) - g(c) + g(c) - g(c - h)}{h} \right) \\
        &= \frac{1}{2} \left( \lim_{h \to 0} \frac{g(c + h) - g(c)}{h} + \lim_{h
        \to 0} \frac{g(c) - g(c - h)}{h} \right) \\
        &= \frac{1}{2} \left( \lim_{h \to 0} \frac{g(c + h) - g(c)}{h} + \lim_{x
        \to c} \frac{g(c) - g(c - (c - x))}{c - x} \right) \\
        &= \frac{1}{2} \left( \lim_{h \to 0} \frac{g(c + h) - g(c)}{h} + \lim_{x
        \to c} \frac{g(c) - g(x)}{c - x} \right) \\
        &= \frac{1}{2} \left( \lim_{h \to 0} \frac{g(c + h) - g(c)}{h} + \lim_{x
        \to c} \frac{g(x) - g(c)}{x - c} \right) \\
        &= \frac{1}{2} (g'(c) + g'(c)) \\
        &= g'(c).
      \end{align*}
  \end{enumerate}
\end{exercise}

\end{document}
