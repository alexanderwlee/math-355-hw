\documentclass{amsart}

\usepackage{enumitem}
\usepackage{mathtools}

\theoremstyle{definition}
\newtheorem*{exercise}{Exercise}

\DeclarePairedDelimiter\abs{\lvert}{\rvert} % absolute value
% Swap the definition of \abs*, so that \abs
% resizes the size of the bars, and the starred version does not.
\makeatletter
\let\oldabs\abs%
\def\abs{\@ifstar{\oldabs}{\oldabs*}}
\makeatother

\newcommand{\N}{\mathbb{N}}
\newcommand{\Z}{\mathbb{Z}}
\newcommand{\Q}{\mathbb{Q}}
\newcommand{\I}{\mathbb{I}}
\newcommand{\R}{\mathbb{R}}
\newcommand{\card}[1]{\abs{#1}}
\newcommand{\st}{\mathrel{:}}

\title{MATH 355: Practice Problems}
\author{Alexander Lee}

\begin{document}

\maketitle

\section*{1 The Real Numbers}

\subsection*{1.2 Some Preliminaries}

\subsection*{1.3 The Axiom of Completeness}

\subsection*{1.4 Consequences of Completeness}

\subsection*{1.5 Cardinality}

\subsection*{1.6 Cantor's Theorem}

\section*{2 Sequences and Series}

\subsection*{2.2 The Limit of a Sequence}

\subsection*{2.3 The Algebraic and Order Limit Theorems}

\begin{exercise}[1]
  \begin{enumerate}[label={(\alph*)}]
    \item Let $\epsilon > 0$ be given. Since $(x_n) \rightarrow 0$, $\exists N
      \in \N$ such that for all $n \ge N$, we have $x_n = \abs{x_n} = \abs{x_n
      - 0} < \epsilon^2$. Hence, $\sqrt{x_n} < \epsilon$. Therefore, for all $n
      \ge N$, we have $\abs{\sqrt{x_n} - 0} = \sqrt{x_n} < \epsilon$.
      Thus, $(\sqrt{x_n}) \rightarrow 0$.
    \item Let $\epsilon > 0$ be given.
  \end{enumerate}
\end{exercise}

\begin{exercise}[2]
\end{exercise}

\begin{exercise}[3]
\end{exercise}

\begin{exercise}[4]
\end{exercise}

\begin{exercise}[7]
\end{exercise}

\begin{exercise}[9]
\end{exercise}

\begin{exercise}[10]
\end{exercise}

\begin{exercise}[12]
\end{exercise}

\subsection*{2.4 The Monotone Convergence Theorem and a First Look at Infinite
Series}

\begin{exercise}[1]
\end{exercise}

\begin{exercise}[3]
\end{exercise}

\begin{exercise}[8]
\end{exercise}

\end{document}
